\documentclass[letterpaper,12pt]{article}
%\documentclass[draft,twoside,letterpaper,12pt]{article}

\title{Jesus Is an Anarchist\\
\Large
\vspace{1em}
A Free-Market, Libertarian Anarchist, That Is---Otherwise What Is Called an Anarcho-Capitalist\footnote{First published at \textless\url{http://anti-state.com}\textgreater\ on December 19, 2001. Herein revised and expanded on \today . Permission to copy, reprint and/or translate this article without the need for request is hereby granted.}}

\author{James Redford\footnote{Email address: \textless\href{mailto:jrredford@yahoo.com}{\nolinkurl{jrredford@yahoo.com}}\textgreater .}}

%\date{October 17, 2011}

\usepackage[ibycus,english]{babel}
\usepackage{cjhebrew}
\usepackage[runin]{abstract}
\usepackage{thinsp}
\usepackage[charter]{mathdesign}
\usepackage[hyphens]{url}
\usepackage{hyperref}
\usepackage{epigraph}
\usepackage{geometry}
\usepackage{appendix}

\urlstyle{sf}

\setlength{\epigraphwidth}{3.67in}

\renewcommand{\sfdefault}{fvs}

\abslabeldelim{\mbox{:\hspace{-0.25em}\quad}}

\newenvironment{squote}
  {\small\quote}
  {\endquote\normalsize}
  
\newenvironment{squotation}
  {\small\quotation}
  {\endquotation\normalsize}
  
\newenvironment{sverse}
  {\small\verse}
  {\endverse\normalsize}

\frenchspacing
\sloppy
%\fussy
\clubpenalty=5000
\widowpenalty=10000

\begin{document}
\maketitle

\renewcommand{\abstractname}{\textsc{Abstract}}
\setlength{\absparindent}{0.5em}
\begin{abstract}
The teachings and actions of Jesus Christ (Yeshua Ha'Mashiach) and the apostles recorded in the New Testament are analyzed in regard to their ethical and political philosophy, with analysis of context vis-\`{a}-vis the Old Testament (Tanakh, or Hebrew Bible) being given. From this analysis, it is shown that Jesus is a libertarian anarchist, i.e., a consistent voluntaryist. The implications this has for the world are profound, and the ramifications of Jesus's anarchism to Christians' attitudes toward government (the state) and its actions are explicated.
\end{abstract}

The above title may seem like strong words, for surely that can't be correct? Jesus an anarchist? One must be joking, right?

But you read correctly, and I will demonstrate exactly that. At this point you may be incredulous, but I assure you that I am quite serious. If you are a Christian and find the above title at all hard to believe then you of all people owe it to yourself to find out what the basis of this charge is, for if the above comes as news to you then you still have much to learn about Jesus and about the most vitally important struggle which has plagued mankind since the dawn of history: mankind's continuing struggle between freedom and slavery, between value-producers and the violent parasitical elite, between peace and war, between truth and deception.

This is the central struggle which defines mankind's history and, sadly, continues to do so. As Christians and as people in general, what we choose to believe and accept as the truth is equally as vitally important, for ultimately it is people's beliefs about the world that will shape and determine what outcomes transpire in the world. If the mass of people believe in political falsehoods and deceptions then mankind will continue to repeat the same gruesome mistakes, as it does presently, and the aforementioned struggle will continue to be no closer to a desirable resolution. Genuine change must first come by changing one's mind, and if what one had believed before was in error then one cannot coherently expect good results to proceed forth from it. And all change starts with the individual. You can help change the world by simply changing your mind. All I ask of you is to believe in the truth---know the truth and the truth will make you free.\footnote{John 8:32.}

It is the purpose of this document to demonstrate the above claim, and if you are a Christian then I submit that it should be your task to honestly consider what is presented here, for if the above claim comes as a surprise then I will show that what you thought you knew about Jesus was not the whole story---Jesus is far more radical than many would have you believe, and for good reason: it threatens the status quo. For the consequence of this truth becoming understood and accepted by even one-tenth of the population would be quite dramatic indeed: governments would topple like so many dominoes. As the 16th century Frenchman \'{E}tienne de La Bo\'{e}tie observed in his \emph{Discourse of Voluntary Servitude},\footnote{\'{E}tienne de La Bo\'{e}tie, \emph{Discours sur la servitude volontaire; ou Contr'un}, likely written in 1552 or 1553, first published in full in 1576. For an English translation by Harry Kurz with an introduction by Murray N. Rothbard, see \emph{The Politics of Obedience: The Discourse of Voluntary Servitude} (New York: Free Life Editions, 1975) \textless\url{http://mises.org/rothbard/boetie.pdf}\textgreater , \textless\url{http://webcitation.org/62VdsStM1}\textgreater .} all governments ultimately rest on the consent of the governed, even totalitarian dictatorships. Now this ``consent'' does not have to be in the form of active promotion and support of the State, it could simply be in the form of hopeless resignation, such as accepting the canard ``nothing's as sure as death and taxes.'' All governments can only exist because the majority---in one form or another---accept them as at least being inevitable. They believe in the deception that even though government may be evil that it is nevertheless a necessary evil, and therefore cannot conceive of a better alternative. But if such were true then Jesus Christ's whole message is a fallacy. But such is not the truth, there is an alternative: \emph{liberty}. And I will show that Jesus has called us to liberty, and that liberty and Christ's message are incompatible with government.

You may wonder where I got the one-tenth figure from in the above if all governments require the acceptance of their rule by the majority of their population. Again, the reason is because this acceptance doesn't have to be active support but merely resigned, as it usually is. If just one-tenth of the population strongly believed that government was itself the greatest moral evil and that there was a better alternative it would be enough to turn the tide. Since most people are followers and uncritically accept the reigning political opinions, those who do not accept the status quo and who are able to form and articulate a critical alternative will come to be the intellectual leaders by default when the popular regime suffers a crisis and people begin to look for alternatives. If the history of governments teach us anything it is that such crisis is a regular occurrence, for governments by their nature tend toward instability. If it be asked \emph{Why then do we still have government?}, it is here answered that it is because no viable alternative to government has been articulated by a critical mass at such a crisis, in that most people throughout history have accepted the deception that government is a necessary evil and could not conceive a better alternative.

Now I will articulate that better alternative, the one that Christ commanded us. I will show that Jesus and His message are necessarily anarchistic. And what better place to start than in the beginning?

\section{Jesus's Very Life Began in an Act of Defiance to Government---and Would Later End in Defiance to Government}
\label{sec:JesusSVeryLifeBeganInAnActOfDefianceToGovernmentAndWouldLaterEndInDefianceToGovernment}

If it were not for Joseph and Mary's intentional act of defying that which they knew to be king Herod the Great's will and escaping with baby Jesus from out of Herod's midst as fugitives to the land of Egypt then Jesus would have been mercilessly killed and needless to say His ministry and the fulfillment of Scripture would have never come about. Thus in the most fundamental of regards, there is a great antagonism from the very start between Jesus and government, to say the least: Jesus was born into the world as a criminal and would latter be killed as a criminal---a criminal as so regarded by the government, that is. And what was baby Jesus's crime? From Matthew 2:1--6 we find the answer:

\begin{squotation}
Now after Jesus was born in Bethlehem of Judea in the days of Herod the king, behold, wise men from the East came to Jerusalem, saying, ``Where is He who has been born King of the Jews? For we have seen His star in the East and have come to worship Him.''

When Herod the king heard this, he was troubled, and all Jerusalem with him. And when he had gathered all the chief priests and scribes of the people together, he inquired of them where the Christ was to be born.

So they said to him, ``In Bethlehem of Judea, for thus it is written by the prophet:

\begin{sverse}
`But you, Bethlehem, in the land of Judah,\\
Are not the least among the rulers of Judah;\\
For out of you shall come a Ruler\\
Who will shepherd My people Israel.'\thinspace''\footnote{New King James Version, as elsewhere unless noted otherwise.}
\end{sverse}
\end{squotation}

So here we learn that Herod became troubled at the thought that there might be someone else who people would come to regard as their king other than Herod. Herod regarded Jesus as a threat to his power---was his fear unjustified? It is my judgement and this document's central thesis that Herod was correct in his assessment of Jesus as being a threat to his power---although not just to Herod as an individual but to all that Herod represents, in a word: \emph{government}; along with the unholy usurpation, deception and subjugation of people that it necessarily entails. For as I will show, Jesus's Kingdom is to be the functional opposite of any Earthbound kingdom which has ever existed. And for government, this is the ultimate crime of which Jesus was guilty, and which required His extermination.

Here we read of this pivotal act of holy defiance to government, without which there would be no Christ as we know of:

\begin{squotation}
Now when they had departed, behold, an angel of the Lord appeared to Joseph in a dream, saying, ``Arise, take the young Child and His mother, flee to Egypt, and stay there until I bring you word; for Herod will seek the young Child to destroy Him.''

When he arose, he took the young Child and His mother by night and departed for Egypt, and was there until the death of Herod, that it might be fulfilled which was spoken by the Lord through the prophet, saying, ``Out of Egypt I called My Son.''\footnote{Matthew 2:13--15.}
\end{squotation}

So enraged was Herod upon learning that the wise men had disobeyed his order to report back to him on the location of baby Jesus that he ordered the extermination of all the male children in Bethlehem and the surrounding areas from age two and younger, all in the hopes that baby Jesus would be among the slaughtered.\footnote{See Matthew 2:12,16--18.} It was only after king Herod the Great had perished that Joseph brought his family out of the land of Egypt, and then only to Nazareth as Herod's son Archelaus was then reigning over Judea.\footnote{See Matthew 2:19--23.}

How very considerate indeed Jesus was being when He advised His disciples the following:

\begin{squote}
Then He charged them, saying, ``Take heed, beware of the leaven of the Pharisees and the leaven of Herod.''\footnote{Mark 8:15.}
\end{squote}

At the time Jesus offered the above advice He would have been referring to Herod Antipas. Jesus would later be mocked and ridiculed by Herod Antipas before finally being put to death as a common criminal by the Roman government.\footnote{See Luke 23:8--12.} In handling the case of Jesus, Herod Antipas asked Jesus many questions, but Jesus refused to answer any of Herod's questions.\footnote{See Luke 23:9.} Thus, not only did Jesus's very life begin in an act of holy defiance to government, but it would also end in holy defiance to government. It was also Herod Antipas who beheaded John the Baptist.\footnote{See Matthew 14:1--12; Mark 6:14--29; Luke 9:7--9.}

The story of Jesus's life can in part be summed up as suffering through this unjust Satanic world system for having preached the Truth, with government being chief among the culprits of this Satanic world system. All one has to do is review the life story of Jesus to plainly see that government---far from being instituted by God---is and has been a demonic tool of Satan used to oppress the righteous. And I will demonstrate that Jesus and the early Church leaders---as recorded the Bible---knew this to be the case and preached the same. The instrument which Satan used in an attempt to snuff-out that Truth in an act of deicide was \emph{government}---from the beginning of Jesus's life to the very end, it was government which sought to exterminate this most dangerous threat of all to its power.

\section{The Golden Rule Unavoidably Results in Anarchism}
\label{sec:TheGoldenRuleUnavoidablyResultsInAnarchism}

Jesus commanded us that in all things we are to treat others as we would want others to treat us. Thus, Jesus said:

\begin{squote}
``Do not think that I came to destroy the Law or the Prophets. I did not come to destroy but to fulfill. For assuredly, I say to you, till heaven and earth pass away, one jot or one tittle will by no means pass from the law till all is fulfilled.''\footnote{Matthew 5:17,18.}
\end{squote}

\begin{squote}
``Therefore, whatever you want men to do to you, do also to them, for this is the Law and the Prophets.''\footnote{Matthew 7:12. See also Luke 6:31.}
\end{squote}

By saying that this commandment is ``the Law and the Prophets,'' Jesus is saying that by following this one commandment that one is thereby fulfilling the Law of Moses and the principles of the Prophets---in other words, Jesus is saying that it is the be-all and end-all when it comes to the proper ethic of social relations. This ultimate social ethic which Jesus commanded everyone to follow is commonly known as the Golden Rule.\footnote{An equivalent formulation of this is love your neighbor as yourself (see Matthew 19:19; 22:36--40; Mark 12:28--34; Luke 10:25--28). Another equivalent formulation of this is Jesus's Commandment that we love one another as He has loved us (see John 15:12,17; 13:15,34,35; 1~John 3:11,12,23; 4:11,20,21). Everything that Jesus ever commanded people to do can be logically reduced back to this one principle.}

But if indeed Jesus actually meant what He said when He spoke these words---and He most certainly did---then this alone is more than enough to prove that Jesus is of necessity an anarchist, and not just any kind of anarchist, but a libertarian, free-market anarchist in particular.

The reason this would necessarily have to be the case is because it is impossible for any actual government to actually abide by the Golden Rule even in theory, let alone in practice. All governments must of necessity violate the Golden Rule, otherwise they would not be governments but would be something else instead.

To understand why this is unalterably true, one must first have a clear and precise understanding of just what a ``government'' is and just what it is not, i.e., the distinguishing characteristics of \emph{government} which differentiates it from all other things that are not governments.

When the word is used in the sense above, \emph{government} (i.e., a \emph{state}) is that organization in society which attempts to maintain, and is generally successful at maintaining, a coercive regional monopoly over ultimate control of the law (i.e., on the courts and police, etc.)---this is a feature of all governments; as well, historically speaking it has always been the case that it is the only organization in society that legally obtains its revenue not by voluntary contribution or payment for contracted services rendered but by coercion.

It is here where we find why it is quite impossible for any government to actually abide by Jesus's ultimate commandment. The reason quite simply is because all governments do to their subjects what they outlaw their subjects to do to them. That is, all governments, in order to be a government, must enforce a coercive monopoly on ultimate control of the law---this is a necessary feature of all governments. All governments set up courts and enforce control over ultimate judicial decision, while outlawing others from engaging in the same practice. Thus, e.g., if a group of people become dissatisfied with the judicial services that the government is providing and decide to set up shop offering their own private arbitration and protection services on the market without obtaining the permission of the government to do so then the government will attack these people and put an end to their competitive judicial services, and would thereby enforce its monopoly on ultimate control over the law. If the government failed to enforce its monopoly on ultimate control over the law then it would cease to be a government, but would instead become just another private protection agency offering its services on a competitive market.

The above scenario leaves out something extremely vital though, as it merely assumes that this government in question somehow obtains its revenue by voluntary contribution and not by coercion. Yet all actual governments throughout history have obtained their revenue \emph{not} by voluntary contribution or payment for contracted services but by coercion. Thus all governments throughout history steal and extort wealth from their subjects which they call ``taxes,'' yet at the same time governments make it illegal for their subjects to steal from each other or from the government. Thus here again in taxes we see that historically all governments do to their subjects what they outlaw their subjects to do to them. I say ``historically'' because while although all governments throughout history have found it necessary to fund their operations through theft and extortion, it is not necessarily the case that all governments in theory must be supported by taxes: one could imagine that most people in a certain society simply voluntarily donate their money to fund a government, as unlikely as that possibility is in practice. So while although a monopoly on ultimate control of the law is a logical necessity of all governments, taxes are not---taxes have simply been a practical necessity throughout all of history in order for governments to function.

And so we find that all governments must of necessity continuously violate Jesus's ultimate social commandment even to simply exist. The principle which all governments are founded upon and follow may properly be termed the \emph{Satanic Principle}.\label{SatanicPrinciple} This logically follows, because to not follow the Golden Rule is to do the opposite of the Golden Rule: i.e., rather than doing to others what you would want others to do to you, you would instead be doing to others what you do \emph{not} want others to do to you. Hence, if we may term the Golden Rule the \emph{Christic Principle}, or otherwise the \emph{Christian Principle}, then it certainly follows that the opposite of this principle would properly be termed the Satanic Principle: which is none other than doing to others what you do \emph{not} want others to do to you.

It is for this reason that anyone who takes Jesus's ultimate ethical commandment seriously must of necessity advocate the abolition of all Earthly governments wherever and whenever they may exist, as governments are necessarily incompatible with Jesus's ultimate ethical commandment and diametrically opposed to it. In passing, it's important for me to distinguish ``Earthly governments'' from what is sometimes called the ``Kingdom of God'' or the ``Kingdom of Christ.'' In the above discussion I have been analyzing governments as they are operated by men here on Earth---but as I will show, the ``Kingdom'' which Christ is to establish on Earth will be the functional and operational opposite of any kingdom which has ever existed on Earth before, i.e., it won't actually be a government in the sense in which I defined above and will in fact be perfectly consistent with the Golden Rule.

Above I also stated that Jesus's commandment of the Golden Rule not only proves that He is an anarchist, but also necessarily a \emph{libertarian}, or free market, anarchist to be specific. The reason that this is so is because an anarchist is simply someone who desires no government to exist: only this and nothing more. Thus, one could desire no government to exist and yet still feel that it is alright to, say, slap people upside the head for no reason. Yet someone who follows the Golden Rule must not do to others what they do not want others to do to them---this necessarily means that one must respect the autonomy of other people's person and their just property: which unavoidably leads to not just anarchism, as was demonstrated above, but also to the free-market, voluntarist, libertarian order.

The rigorous proof of this is that everyone, by definition, objects to others aggressing against what they regard as their own property. If such were not the case then, by definition, such action would not be an aggression but a voluntary action. But ultimately all just property titles can (1) be traced back by way of voluntary transactions (which would thus be consistent with the Golden Rule) to the homesteading of unused resources; or (2) in the case in which such resources were expropriated from (or abandoned by) a just owner and the just owner or his heir(s) can no longer be identified or are deceased, where the first nonaggressor possesses the resource (which can then be considered another form of homesteading). Thus, for anyone to come into possession of property which either was not homesteaded by themselves or which was not obtained by a voluntary transaction would thereby be violating the Golden Rule, for to do so would mean that they are obtaining a good by involuntary means from another who can trace their possession of the resource either to direct homesteading or through voluntary transactions leading back to homesteading (i.e., of either of the two types given above). Yet, by definition, this aggressor would not want others to take his property against his will which he had come into possession of by voluntary means---and surely everyone possesses such property, even if it is just their own body.

Hence, if Jesus was serious about the Golden Rule---and He certainly was---then it necessarily means that He is a consistent libertarian, as the Golden Rule as a political ethic is completely congruent with the libertarian Nonaggression Principle, i.e., that no person or group of people may initiate the use of force against another, or threaten to initiate force against another.

\section{Jesus Does Not Respect the Person of Men}
\label{sec:JesusDoesNotRespectThePersonOfMen}

According to the Bible, every person is equally subject to the commands of God, and one does not become exempt from God's law simply because one has managed to receive some sort of title of nobility, office or rank. We are instructed to treat everyone by the same law. Yet this automatically rules out the possibility that governments could ever be legitimate, as they can only exist do to a privilege of monopoly on the ultimate control over the law which they enforce while excluding all competitors. As well, they collect taxes, which they call ``theft'' and ``extortion'' if anyone else engages in the same behavior against them or others. 

As it is recorded in the Gospels, it seems that the people who knew of Jesus in His day were aware that He did not regard the person of men (i.e., titles of nobility, office or rank, etc.):

\begin{squote}
And they sent to Him their disciples with the Herodians, saying, ``Teacher, we know that You are true, and teach the way of God in truth; nor do You care about anyone, for You do not regard the person of men.''\footnote{Matthew 22:16. See also Mark 12:14.}
\end{squote}

Yet this would have been merely conforming to people's expectation that Jesus would have been following the Old Testament commands not to regard the person of men.\footnote{See Leviticus 19:15; Deuteronomy 1:17; 16:19; Job 32:21; 34:19; Proverbs 28:21.} But that this is indeed the case was confirmed in the apostles' writings. Paul writes:

\begin{squote}
But from those who seemed to be something---whatever they were, it makes no difference to me; God shows personal favoritism to no man---for those who seemed to be something added nothing to me.\footnote{Galatians 2:6.}
\end{squote}

And as James writes:

\begin{squote}
If you really fulfill the royal law according to the Scripture, ``You shall love your neighbor as yourself,'' you do well; but if you show partiality, you commit sin, and are convicted by the law as transgressors.\footnote{James 2:8,9. See also 1~Peter 1:16.}
\end{squote}

Yet consider what James's above admonition means as it concerns Jesus's ultimate ethical command of the Golden Rule.\footnote{See Matthew 7:12; Luke 6:31.} If we as Christians were to take Jesus's command seriously and apply it to everyone without partiality, then it would necessarily require that we demand the abolition of all governments wherever they may exist, as they can only exist by a continuous violation of the Golden Rule.\footnote{See Section \ref{sec:TheGoldenRuleUnavoidablyResultsInAnarchism} on page \pageref{sec:TheGoldenRuleUnavoidablyResultsInAnarchism} of this article.}

\section{Jesus on Taxes: Nothing Is (Rightly) Caesar's!}
\label{sec:JesusOnTaxesNothingIsRightlyCaesarS}

The story of Jesus commanding us to give unto Caesar that which is Caesar's\footnote{See Matthew 22:15--22; Mark 12:13--17; Luke 20:20--26.} is commonly misrepresented as His commanding us to give to Caesar the denari which he asks for (i.e., to pay taxes to government), as---it is assumed---the denari are Caesar's, being that they have Caesar's image and name on them. But Jesus never said that this was so! What Jesus did say though was an ingenious case of rhetorical misdirection to avoid being immediately arrested, which would have interfered with Old Testament prophecy of His betrayal as well as His own previous predictions of betrayal.

When the Pharisees asked Jesus whether or not it is lawful to pay taxes to Caesar, they did so as a ruse in the hopes of being able to either have Him arrested as a rebel by the Roman authorities or to have Him discredited in the eyes of His followers. At this time in Israel's history it was an occupied territory of the Roman Empire, and taxes---which were being used to support this occupation---were much-hated by the mass of the common Jews. Thus, this question was a clever Catch-22 posed to Jesus by the Pharisees: if Jesus answered that it is \emph{not} lawful then the Pharisees would have Him put away, but if He answered that it \emph{is} lawful then He would appear to be supporting the subjection of the Jewish people by a foreign power. Luke 20:20 makes the Pharisees' intent in asking this question quite clear:

\begin{squote}
So they watched Him, and sent spies who pretended to be righteous, that they might seize on His words, in order to deliver Him to the power and the authority of the governor.
\end{squote}

Thus, Jesus was not free to answer in just any casual manner. Of the Scripture prophecies which would have gone unfulfilled had He answered that it was fine to decline paying taxes and been arrested because of it are the betrayal by Judas\footnote{See Psalm 41:9; Zechariah 11:12,13.} and Jesus's betrayer replaced.\footnote{See Psalm 109:8; Acts 1:20. See also Acts 1:15--26; Psalm 69:25.} Here is a quote from Peter on this matter from Acts 1:16:

\begin{squote}
``Men and brethren, this Scripture had to be fulfilled, which the Holy Spirit spoke before by the mouth of David concerning Judas, who became a guide to those who arrested Jesus.''
\end{squote}

As recorded in Matthew 26:54,56 and Mark 14:49, Jesus testifies to this exact same thing after He was betrayed by Judas. As well, Jesus Himself twice foretold of His betrayal before He was asked the question on taxes.\footnote{See Matthew 17:22; 20:18; Mark 9:31; 10:33; Luke 9:44; 19:31.} See also John 13:18--30, which testifies to the necessity of the fulfillment of Psalm 41:9, as Jesus here foretells of His betrayal by Judas.

In addition, it appears that the only reason Jesus paid the temple tax\label{TempleTax} (and by supernatural means at that) as told in Matthew 17:24--27 was so as not to stir up trouble which would have interfered with the fulfillment of Old Testament Scripture and Jesus's previous prediction of His betrayal as told in Matthew 17:22---neither of which would have been fulfilled had Jesus not paid the tax and been arrested because of it. Jesus Himself supports this view when He said of it ``Nevertheless, lest we offend \mbox{them \ldots,''} which can also be translated ``But we don't want to cause trouble.''\footnote{Contemporary English Version.} He said this after in effect saying that those who pay customs and taxes are not free\footnote{See Matthew 17:25,26.}---yet one reason Jesus came was to call us to liberty.\footnote{See Luke 4:18; Galatians 4:7; 5:1,13,14; 1~Corinthians 7:23; 15:23,24; 2~Corinthians 3:17; James 1:25; 2:12.}

It should be remembered in all of this that it was Jesus Himself who told us, ``Behold, I send you out as sheep in the midst of wolves. Therefore be wise as serpents and harmless as doves.''\footnote{Matthew 10:16.} Jesus was being wise as a serpent as He never told us to pay taxes to Caesar, of which He could have done and still fulfilled Scripture and His previous predictions of betrayal. But the one thing He couldn't have told people was that it was okay not to pay taxes as He would have been arrested on the spot, and Scripture and His predictions of betrayal would have gone unfulfilled. Yet the most important thing in all this is what Jesus did not say. Jesus never said that all or any of the denari were Caesar's! Jesus simply said ``Give to Caesar that which is Caesar's.'' But this just begs the question, \emph{What is Caesar's?} Simply because the denari have Caesar's name and image on them no more make them his than one carving their name into the back of a stolen T.V. set makes it theirs. Yet everything Caesar has has been taken by theft and extortion, therefore nothing is rightly his.

\section{Tax Collection Is a Sin!}
\label{sec:TaxCollectionIsASin}

A further demonstration that Jesus considers the institution of taxation to be unjust is given in the below story:

\begin{squotation}
As Jesus passed on from there, He saw a man named Matthew sitting at the tax office. And He said to him, ``Follow Me.'' So he arose and followed Him.

Now it happened, as Jesus sat at the table in the house, that behold, many tax collectors and sinners came and sat down with Him and His disciples. And when the Pharisees saw it, they said to His disciples, ``Why does your Teacher eat with tax collectors and sinners?''

When Jesus heard that, He said to them, ``Those who are well have no need of a physician, but those who are sick. But go and learn what this means: `I desire mercy and not sacrifice.' For I did not come to call the righteous, but sinners, to repentance.''\footnote{Matthew 9:9--13. See also Mark 2:14--17; Luke 5:27--32.}
\end{squotation}

It's important to point out here that Jesus actually made a stronger case against the unrighteousness of tax collectors than the Pharisees originally had in questioning Jesus's disciples about it: the Pharisees actually separated the tax collectors from the sinners when they asked ``Why does your Teacher eat with tax collectors and sinners?'' Yet when Jesus heard this He answered the Pharisees by lumping the two groups together under the category of sinners---thus: ``For I did not come to call the righteous, but sinners, to repentance.''

Yet since this is the story of Matthew the tax collector being called to repentance by Jesus we will do well to ask how it was that Matthew obtained repentance. The answer: by first giving up tax-collecting! And from this beginning Matthew would thus go on to become one of Jesus's twelve disciples.

It may be pointed out in response that ``all have sinned and fall short of the glory of God.''\footnote{Romans 3:23.} But the below passages make clear just how unrighteous and iniquitous an occupation Jesus considers tax collection to be:

\begin{squote}
``For if you love those who love you, what reward have you? Do not even the tax collectors do the same? And if you greet your brethren only, what do you do more than others? Do not even the tax collectors do so?''\footnote{Matthew 5:46,47.}
\end{squote}

\begin{squote}
``And if he refuses to hear them, tell it to the church. But if he refuses even to hear the church, let him be to you like a heathen and a tax collector.''\footnote{Matthew 18:17.}
\end{squote}

\section{On Paul, Romans 13 and Titus 3:1}
\label{sec:OnPaulRomans13AndTitus31}

It is often claimed that Christians are required to submit to government, as this is supposedly what Paul commanded that we are supposed to do in Romans 13. Thus, Paul writes:

\begin{squote}
Let every soul be subject to the governing authorities. For there is no authority except from God, and the authorities that exist are appointed by God. Therefore whoever resists the authority resists the ordinance of God, and those who resist will bring judgment on themselves. For rulers are not a terror to good works, but to evil. Do you want to be unafraid of the authority? Do what is good, and you will have praise from the same. For he is God's minister to you for good. But if you do evil, be afraid; for he does not bear the sword in vain; for he is God's minister, an avenger to execute wrath on him who practices evil. Therefore you must be subject, not only because of wrath but also for conscience' sake. For because of this you also pay taxes, for they are God's ministers attending continually to this very thing. Render therefore to all their due: taxes to whom taxes are due, customs to whom customs, fear to whom fear, honor to whom honor.\footnote{Romans 13:1--7.}
\end{squote}

But in actual fact Paul never does tell us in the above excerpt from Romans 13 to submit to government!---at least certainly not as they have existed on Earth and are operated by men. In fact, Paul would be an outright, barefaced hypocrite were he to command anyone to do such a thing: for Paul himself did not submit to government, and if he had then he would not even have been alive to be able to write Romans 13. It is quite a good thing that Paul did disobey government, as we would not even know of a Paul in the Bible had he not disobeyed government. As when Paul was still known as Saul of Tarsus he escaped from the city of Damascus as he knew that the governor of that city, acting under the authority of Aretas the king, was coming with a garrison to arrest him in order that he be executed. This was right after Saul's conversion to Jesus Christ on the road to Damascus. The Jews in Damascus, hearing of Saul's conversion, plotted to kill him as a traitor to their cause in persecuting the Christians. Saul was let out of a window in the wall of Damascus under cover of night by some fellow disciples in Christ.\footnote{See Acts 9:23--25.} In none of Paul's later writings does he dispraise, or disassociate himself from, these actions that he took in knowingly and purposely disobeying government: in fact, this very event is one of the things that he later cites in demonstration of his unwavering commitment to Christ!\footnote{See 2~Corinthians 11:22--33.}

Indeed, ever since Paul's conversion to Jesus Christ, he spent the rest of his entire life in rebellion against mortal governments, and would at last---just as with Jesus before him---be executed by government, in this case by having his head chopped off. Paul was continuously in and out of prisons throughout his entire ministry for preaching the gospel of Christ; he was on five separate occasions lashed with stripes 39 times each by the ``authorities'' for preaching Christ; he was beaten with rods by the ``authorities'' for preaching Christ; and none of these rebellions of his did he ever reprehend: indeed he cited them all as evidence of his commitment to Jesus!\footnote{Ibid.}

But even more importantly, if Paul is saying in Romans 13 what many people have said he meant, i.e., that people should obey mortal, Earthly governments, then it is questionable whether Paul could even be a genuine Christian. For as was pointed out above, Jesus would not even have existed as we know of today had it not been for Joseph and Mary intentionally disobeying king Herod the Great and escaping from his reach when they knew that Herod desired to destroy baby Jesus.\footnote{See Matthew 2:13,14.} Thus, if indeed Paul meant in Romans 13 that we are to obey Earthly governments then this would mean that Paul would rather have Joseph and Mary obey king Herod the Great and turn baby Jesus over to be killed.

So what in the world is going on here with Paul and Romans 13? Is Paul a hypocrite? Is Paul being contradictory? Actually, \emph{No} to both. Once again, as with Jesus's answer to the question on taxes, this is another ingenious case of rhetorical misdirection. Paul was counting on the fact that most people who would be hostile to the Christian church---the Roman ``authorities'' in particular---would, upon reading Romans 13, naturally interpret it from the point of view of legal positivism: i.e., that such people would take for granted that the ``governing authorities'' and ``rulers'' spoken of must refer to the men who operate the governments on Earth. But never does Paul anywhere say that this is so! (\emph{Legal positivism} is the doctrine that whichever gang is best able to overpower others with arms and might and thereby subjugate the populace and who then proceed to proclaim themselves the ``authority'' are on that account the rightful ``Authority.'')

But before proceeding with the above analysis, what would the motive be for Paul to include such rhetorical misdirection in his letter to the people at the church of Rome? In answering this, it must be remembered that just as with Jesus, Paul was not free to say just anything that he wanted. The early Christians were a persecuted minority under the close surveillance of the Roman government as a possible threat to its power. Here is Biblical proof of this assertion written by Paul himself:

\begin{squote}
And this occurred because of false brethren secretly brought in (who came in by stealth to spy out our liberty which we have in Christ Jesus, that they might bring us into bondage), to whom we did not yield submission even for an hour, that the truth of the gospel might continue with you.\footnote{Galatians 2:4,5.}
\end{squote}

Paul never intended that his letter to the Roman church be kept secret, and he knew that it would be copied and distributed amongst the populace, and thus inevitably it would fall into the hands of the Roman government, especially considering that this letter was going directly into the belly of the beast itself: the city of Rome. Thus by including this in the letter to the church at Rome he would help put at ease the fears of the Roman government so that the persecution of the Christians would not be as severe and so that the more important task of the Church, that of saving people's souls, could more easily continue unimpeded. But Paul wrote it in such a way that a truly knowledgeable Christian at the time would have no doubt as to what was actually meant.

The Church leaders at the time would have known that Paul obviously couldn't have meant the people who control the mortal governments as they exist on Earth when he referred to the ``governing authorities'' and ``rulers'' in Romans 13, for that would have made Paul a shameless hypocrite and also meant that he would desire that baby Jesus had been killed (for surely the histories of Paul and Jesus's lives would have been fresh on their minds). The only answer that can make any sense of this seeming riddle is that one doesn't actually become a true ``governing authority'' or ``ruler'' simply because one has managed by way of deception, terror, murder and might to subjugate a certain population and then proceed to thereby proclaim oneself the ``King'' or the ``Authority'' or the ``Ruler.'' Instead, what Paul is saying is that the only \emph{true} and \emph{real} authorities are \emph{only} those who God appoints, i.e., one cannot become a \emph{real} authority or ruler in the eyes of God simply because through force of arms one has managed to subjugate a population and then proclaim oneself the potentate. Thus, by saying this Paul was actually rebuking the supposed authority of the mortal governments as they exist on Earth and are operated by men!

``Let every soul be subject to the governing authorities. For there is no authority except from God, and the authorities that exist are appointed by God''\footnote{Romans 13:1.} leaves wide open the possibility that those who control the mortal governments on Earth are not true authorities as appointed by God. The fallacy most people make when encountering a statement such as this is to unthinkingly and automatically assume that Paul must be referring to the people in control of the mortal governments that exist on Earth---for after all, don't these people who run these Earthly governments call themselves the ``governing authorities''? Do they not teach their subjects from birth that they are the ``rulers'' and the ``authorities''? But when we factor in the life histories of both Jesus and Paul, then it can leave no room for doubt: Paul most certainly could not have been referring in Romans 13 to the people who control the mortal governments as they exist on Earth---otherwise Paul would be an outright hypocrite as well as an advocate of deicide against baby Jesus. Indeed, God Himself directly confirms this very thing as He spoke to Hosea: ``They set up kings, but not by Me;~/ They made princes, but I did not acknowledge them.''\footnote{Hosea 8:4.}

Let us continue with this analysis as it specifically concerns Romans 13:3,4:

\begin{squote}
For rulers are not a terror to good works, but to evil. Do you want to be unafraid of the authority? Do what is good, and you will have praise from the same. For he is God's minister to you for good. But if you do evil, be afraid; for he does not bear the sword in vain; for he is God's minister, an avenger to execute wrath on him who practices evil.
\end{squote}

Here Paul uses deep Christian references to logically code his necessarily actual message, for Jesus Christ said that all who bear the sword do indeed bear it in vain.\footnote{See Matthew 26:52; Revelation 13:10.} So why is Paul seemingly here contradicting Jesus Christ's own teachings on this matter? In order to reconcile the apparent contradiction and hence to comprehend what Paul is in actuality saying here requires a firm understanding of early Christian terminology, such as used by Jesus and the original Church fathers. Paul is not talking about a literal sword, i.e., actual \emph{physical force}, such as used by all the Earthly, mortal governments. Paul is talking about the \emph{Word of God},\footnote{See Ephesians 6:17.} which is the sword that Jesus Christ bears,\footnote{See Matthew 10:34; Hebrews 4:12; Revelation 1:16; 19:15,21.} and which figurative sword is none other than simply \emph{the truth}. This is the only ``sword'' not borne in vain. This is also the figurative ``fire'' that Jesus came to send to the Earth\footnote{See Matthew 3:11; Luke 3:16; 12:49. See also Matthew 10:34.}---that figurative ``fire'' being the Word of God, i.e., \emph{the truth}.

As Paul wrote in his letters concerning the pretended ``authorities'' of the mortal, Earthly governments:

\begin{squote}
However, we speak wisdom among those who are mature, yet not the wisdom of this age, nor of the rulers of this age, who are coming to nothing. But we speak the wisdom of God in a mystery, the hidden wisdom which God ordained before the ages for our glory, which none of the rulers of this age knew; for had they known, they would not have crucified the Lord of glory.\footnote{1~Corinthians 2:6--8.}
\end{squote}

Further,

\begin{squote}
Remember that Jesus Christ, of the seed of David, was raised from the dead according to my gospel, for which I suffer trouble as an evildoer, even to the point of chains; but the word of God is not chained.\footnote{2~Timothy 2:8,9.}
\end{squote}

\begin{squote}
Yes, and all who desire to live godly in Christ Jesus will suffer persecution.\footnote{2~Timothy 3:12.}
\end{squote}

What the above passages clearly demonstrate is that Paul certainly did indeed think that the mortal, Earthly rulers were a terror to good works. Paul even wrote that ``the rulers of this age \ldots are coming to nothing''!

Paul elsewhere wrote that the \emph{only} genuine potentate is Jesus Christ, saying that Jesus ``is the blessed and only Potentate, the King of kings and Lord of lords.''\footnote{1~Timothy 6:15. See also 1~Timothy 1:17; Acts 17:6,7; James 4:12.} But as \emph{true} Christians, being members in the body of Christ,\footnote{See 1~Corinthians 12:12--27; Romans 12:5; Ephesians 5:30.} we are co-potentates along with Jesus\footnote{See 1~Corinthians 6:1--8; Luke 12:57; 22:30; Revelation 5:10; Daniel 7:27.}---but \emph{only} insofar as we remain within the Spirit of Christ. If were were to pick up and use a literal sword, i.e., if we were to use actual aggressive physical force such as the mortal, Earthly governments do, then we would be doing so in vain,\footnote{See Matthew 26:52; Revelation 13:10.} and would no longer be acting under the authority of Jesus as the \emph{only true} potentate. In other words, we are to speak the hard, hated and dangerous truth, such as regarding the inherently diabolical nature of government. This is our sword, and it is the only sword which is not borne in vain.

But, some may inquire, what about Paul telling us to pay taxes in Romans 13:6--7? Thus, Paul wrote:

\begin{squote}
For because of this you also pay taxes, for they are God's ministers attending continually to this very thing. Render therefore to all their due: taxes to whom taxes are due, customs to whom customs, fear to whom fear, honor to whom honor.\footnote{Romans 13:6,7.}
\end{squote}

But does Paul really tell us to pay taxes here? Again, just as with Jesus, nowhere does Paul actually tell anyone to pay any taxes! Paul continues with the rhetorical misdirection that he started in the beginning of Romans 13, knowing---just as Jesus knew before him---that those who would be hostile to the Christian church would automatically assume what they are predisposed to assume: i.e., that the taxes and customs ``due'' are due to those in control of the governments who levy them. But here Paul was being wise as a serpent and harmless as a dove, as Paul never said any such thing. For when Paul says ``Render therefore to all their due: taxes to whom taxes are due, customs to whom customs'' this just begs the question: \emph{To whom are taxes and customs due?} The answer to which could quite possibly be \emph{``No one.''} And this is precisely how Paul proceeds to answer his own question-begging statement, in Romans 13:8--10:

\begin{squote}
Owe no one anything except to love one another, for he who loves another has fulfilled the law. For the commandments, ``You shall not commit adultery,'' ``You shall not murder,'' ``You shall not steal,'' ``You shall not bear false witness,'' ``You shall not covet,'' and if there is any other commandment, are all summed up in this saying, namely, ``You shall love your neighbor as yourself.'' Love does no harm to a neighbor; therefore love is the fulfillment of the law.
\end{squote}

So there we have it in no uncertain terms: owe no one anything except to love one another! Yet since when have taxes ever had the slightest thing to do with love? As was explained above, all mortal governments throughout history steal and extort wealth from their subjects which they call ``taxes,'' yet at the same time governments make it illegal for their subjects to steal from each other or from the government. Thus in taxes we see that historically all governments do to their subjects what they outlaw their subjects to do to them. Thus, all Earthly, mortal governments, by levying taxes, break the Golden Rule which Jesus commanded everyone as the supreme law.

In the earlier discussion on Jesus and taxes we learned that when Jesus said ``Give on to Caesar that which is Caesar's and give unto the Lord that which is the Lord's'' he was, in effect, actually saying that one need not give anything to Caesar: as nothing is rightly his, considering that everything that Caesar has has been taken by theft and extortion.

And what of Paul writing in Titus 3:1: ``Remind them to be subject to rulers and authorities, to obey, to be ready for every good work''? As was clearly demonstrated above, Paul was referring to the \emph{true} higher authorities as recognized by God, not to the diabolical, Satanic, mortal governments as they have existed on Earth---as Paul spent his entire ministry in rebellion against the Earthbound, mortal ``authorities,'' and was at last put to death by them.\footnote{For other cases of righteous disobedience to government recorded in the Bible, see Exodus 1:15--2:3; 1~Samuel 19:10--18; Esther 4:16; Daniel 3:12--18; 6:10; Matthew 2:12,13; Luke 23:8,9; Acts 5:29; 9:25; 17:6--8; 2~Corinthians 11:32,33.}

And as further proof of this, consider Paul's advice to Christians as regarding being judged by what the government considers the ``authority'':

\begin{squotation}
Dare any of you, having a matter against another, go to law before the unrighteous, and not before the saints? Do you not know that the saints will judge the world? And if the world will be judged by you, are you unworthy to judge the smallest matters? Do you not know that we shall judge angels? How much more, things that pertain to this life? If then you have judgments concerning things pertaining to this life, do you appoint those who are least esteemed by the church to judge? I say this to your shame. Is it so, that there is not a wise man among you, not even one, who will be able to judge between his brethren? But brother goes to law against brother, and that before unbelievers!

Now therefore, it is already an utter failure for you that you go to law against one another. Why do you not rather accept wrong? Why do you not rather let yourselves be cheated? No, you yourselves do wrong and cheat, and you do these things to your brethren!\footnote{1~Corinthians 6:1--8.}
\end{squotation}

Paul said that the government judges ``are least esteemed by the church to judge''! It is clear that he considered them to be no authority at all!

But moreover, even Jesus didn't consider the Earthly, mortal ``rulers'' to be true rulers and authorities! Thus, as Mark records:

\begin{squote}
But Jesus called them to Himself and said to them, ``You know that those who are considered rulers over the Gentiles lord it over them, and their great ones exercise authority over them. Yet it shall not be so among you; but whoever desires to become great among you shall be your servant. And whoever of you desires to be first shall be slave of all. For even the Son of Man did not come to be served, but to serve, and to give His life a ransom for many.''\footnote{Mark 10:42--45. See also Matthew 18:4; 20:25--28; Mark 9:35; Luke 22:25,26.}
\end{squote}

By saying this Jesus was in fact rebuking the supposed ``authority'' of the Earthly ``rulers''! Just because mortals on Earth may consider someone to be an ``authority'' and ``ruler'' does not mean that God considers them to be so!

\section{On Peter and 1~Peter 2:13--18}
\label{sec:OnPeterAnd1Peter21318}

Another Bible passage that is sometimes cited by statists in an attempt to demonstrate that people ought to submit to mortal government is 1~Peter 2:13--18:

\begin{squotation}
Therefore submit yourselves to every ordinance of man for the Lord's sake, whether to the king as supreme, or to governors, as to those who are sent by him for the punishment of evildoers and for the praise of those who do good. For this is the will of God, that by doing good you may put to silence the ignorance of foolish men---as free, yet not using liberty as a cloak for vice, but as bondservants of God. Honor all people. Love the brotherhood. Fear God. Honor the king.

Servants, be submissive to your masters with all fear, not only to the good and gentle, but also to the harsh.
\end{squotation}

But Peter himself did not so submit! Peter and the apostles were arrested in Jerusalem by the Sadducees for preaching the gospel of Jesus and brought before the Sanhedrin court for questioning:

\begin{squotation}
And when they had brought them, they set them before the council. And the high priest asked them, saying, ``Did we not strictly command you not to teach in this name? And look, you have filled Jerusalem with your doctrine, and intend to bring this Man's blood on us!''

But Peter and the other apostles answered and said: ``We ought to obey God rather than men. The God of our fathers raised up Jesus whom you murdered by hanging on a tree. Him God has exalted to His right hand to be Prince and Savior, to give repentance to Israel and forgiveness of sins. And we are His witnesses to these things, and so also is the Holy Spirit whom God has given to those who obey Him.''\footnote{Acts 5:27--32.}
\end{squotation}

So here we have it from Peter himself: We ought to obey God rather than men! Yet Jesus already commanded that the ultimate Law is for everyone to treat others as they themselves would want to be treated---therefore, according to Peter, any commands by men that are contrary to this ultimate Law are automatically null and void.

Once again one must consider that the Christians at this time were a persecuted minority under the surveillance of the mortal ``authorities'' as possible insurrectionists, and so statements like what is written in 1~Peter 2:13--18 were included to pacify such ``authorities'' so that the most important task of saving people's souls could continue---yet, just as Paul included an ``escape clause'' in Romans 13 (``Owe no one anything except to love one another''), Peter also includes an escape clause contained in 1~Peter 2:13--18, which is the 16th verse therein:

\begin{squote}
For this is the will of God, that by doing good you may put to silence the ignorance of foolish men---\textsuperscript{verse~16}as free, yet not using liberty as a cloak for vice, but as bondservants of God.
\end{squote}

The New International Version Bible translates verse 16 as ``Live as free men, but do not use your freedom as a cover-up for evil; live as servants of God.'' Most other modern English Bible versions translate the beginning of this passage as either ``Live as free'' or ``Act as free.'' So in other words, when this is combined with what Peter said in Acts 5:29, we can take the entire passage of 1~Peter 2:13--18 to mean that we ought to obey all the ordinances of men: except for all such ordinances that happen to conflict with our God-given liberty and Jesus's ultimate commandment---which is virtually every single one of them! But other than that, do indeed obey every other ordinance of man, for in so doing one will merely be obeying Jesus's commandment---in which case the ordinances of man which one ought to obey are merely redundant!

Also, consider the following statement by Peter which some etatists might try to construe in their favor:

\begin{squote}
then the Lord knows how to deliver the godly out of temptations and to reserve the unjust under punishment for the day of judgment, and especially those who walk according to the flesh in the lust of uncleanness and despise authority. They are presumptuous, self-willed. They are not afraid to speak evil of dignitarie\mbox{s \ldots}\footnote{2~Peter 2:9,10.}
\end{squote}

As has already been pointed out, the statist fallacy when encountering such statements is to automatically deem the ``authorities'' and ``dignitaries'' spoken of in these cases as necessarily being the ``authorities'' and ``dignitaries'' that the positive law (i.e., the government's law) so regards---but such cannot be the actual case, as it is written by Hosea, as spoken to him by God: ``They set up kings, but not by Me;~/ They made princes, but I did not acknowledge them.''\footnote{Hosea 8:4.}

As well, Jesus Himself rebuked the supposed ``authority'' of the Earthly ``rulers'':

\begin{squote}
But Jesus called them to Himself and said to them, ``You know that those who are considered rulers over the Gentiles lord it over them, and their great ones exercise authority over them. Yet it shall not be so among you; but whoever desires to become great among you shall be your servant. And whoever of you desires to be first shall be slave of all. For even the Son of Man did not come to be served, but to serve, and to give His life a ransom for many.''\footnote{Mark 10:42--45. See also Matthew 18:4; 20:25--28; Mark 9:35; Luke 22:25,26.}
\end{squote}

\section{The Ruler and God of This World and Age Which All Mortal Governments Worship is Satan}
\label{sec:TheRulerAndGodOfThisWorldAndAgeWhichAllMortalGovernmentsWorshipIsSatan}

The Bible is quite explicit as to whom it is that really controls all the mortal governments on Earth, and which god is the god that the mortal rulers worship:

\begin{squotation}
Then the devil, taking Him up on a high mountain, showed Him all the kingdoms of the world in a moment of time. And the devil said to Him, ``All this authority I will give You, and their glory; for this has been delivered to me, and I give it to whomever I wish. Therefore, if You will worship before me, all will be Yours.''

And Jesus answered and said to him, ``Get behind Me, Satan! For it is written, `You shall worship the Lord your God, and Him only you shall serve.'\thinspace''\footnote{Luke 4:5--8. See also Matthew 4:1--11; Mark 1:12,13; Luke 4:1--13.}
\end{squotation}

This is one of the offers Satan made to Christ during the forty days in which Satan tempted Jesus, an event now sometimes referred to as the Temptation of Christ. Satan wasn't lying when he made the above offer to Jesus: it was an absolutely real offer that Satan would have delivered on. This is necessarily the case, as Luke even writes in verse 2 of the above chapter that here Jesus was ``tempted for forty days by the devil''---thus, this had to be a real offer or else it could hardly qualify as a real temptation, as certainly Jesus would have known whether or not what Satan said here was true: if what Satan was saying here were false then Jesus would have already known that, and hence Satan's offer could not have been the least bit tempting to Jesus.

How true indeed Satan was being when he said that all the kingdoms of the world have been delivered to him, and that he gives them to whomever he wishes: which are those who worship him as their god! All Earthly, mortal potentates have quite literally made a pact with Satan! Every last one of them has literally sold their soul to Satan in return for Earthly power! As God spoke as recorded in Hosea 8:4: ``They set up kings, but not by Me;~/ They made princes, but I did not acknowledge them.''

Jesus later said on three separate occasions that Satan is the ruler of this world:

\begin{squote}
``Now is the judgment of this world; now the ruler of this world will be cast out.''\footnote{John 12:31.}
\end{squote}

\begin{squote}
``I will no longer talk much with you, for the ruler of this world is coming, and he has nothing in Me.''\footnote{John 14:30.}
\end{squote}

\begin{squote}
``And when He has come, He will convict the world of sin, and of righteousness, and of judgment: of sin, because they do not believe in Me; of righteousness, because I go to My Father and you see Me no more; of judgment, because the ruler of this world is judged.''\footnote{John 16:8--11.}
\end{squote}

Additionally, Paul in two separate letters writes that Satan is the god and ruler of this age:

\begin{squote}
But even if our gospel is veiled, it is veiled to those who are perishing, whose minds the god of this age has blinded, who do not believe, lest the light of the gospel of the glory of Christ, who is the image of God, should shine on them.\footnote{2~Corinthians 4:3,4.}
\end{squote}

And

\begin{squote}
Put on the whole armor of God, that you may be able to stand against the wiles of the devil. For we do not wrestle against flesh and blood, but against principalities, against powers, against the rulers of the darkness of this age, against spiritual hosts of wickedness in the heavenly places.\footnote{Ephesians 6:11,12.}
\end{squote}

All one has to do to realize just how literal and true Satan, Jesus and Paul were being when they made the above statements\footnote{The present epoch which we live in could accurately be designated the \emph{Satanic Age}, or the \emph{Age of Satan's Earthly Rulership}. The Earthly, mortal governments are simply the outward, physical mechanism of Satan's current rulership over the people of Earth.} is to consider that more than six times the amount of noncombatants have been systematically murdered for purely ideological reasons by their own governments within the past century than were killed in that same time-span from wars. From 1900 to 1923, various Turkish regimes murdered from 3.5 million to over 4.3 million of its own Armenians, Greeks, Nestorians, and other Christians. The Soviet government murdered over 61 million of its own noncombatant subjects. The communist Chinese government murdered over 76 million of it own subjects. The National Socialist German government murdered some 16 million of it own subjects. And that's only a sampling of governments mass-murdering their own noncombatant subjects within the past century.\footnote{The preceding figures are from Prof. Rudolph Joseph Rummel's University of Hawaii website at \textless\url{http://hawaii.edu/powerkills/}\textgreater .} Over 800,000 Christian Tutsis in Rwanda were hacked to death with machetes between April and July of 1994 by the Hutu-led military force after the Tutsis had been disarmed by governmental decree in the early 1990s, of which disarmament decree the United Nations helped to enforce. On several occasions, United Nations soldiers stationed in Rwanda actually handed over helpless Tutsi Christians under their protection to members of the Hutu military. They then stood by as their screaming charges were unceremoniously hacked to pieces. This massacre was facilitated by a national I.D. card system which the Hutus used to track down and identify the Christian Tutsis.\footnote{For the previous information on the Rwandan genocide, see the following articles and book: Karen MacGregor, ``Survivors sue UN for `complicity' in Rwanda genocide,'' \emph{The Independent} (U.K.), January 11, 2000. ``UN chief helped Rwanda killers arm themselves,'' \emph{The Observer} (U.K.), September 3, 2000. ``ID Cards Became Death Certificates During Genocide, Says Expert,'' \emph{Hirondelle News Agency}, March 1, 2006. Peter Hammond, \emph{Holocaust in Rwanda: The Roles of Gun Control, Media Manipulation, Liberal Church Leaders and the United Nations} (Cape Town, South Africa: Frontline Fellowship, 1996). Missionary Rev. Dr. Peter Hammond, Founder and Director of Frontline Fellowship, arrived in Rwanda before the bodies had been buried and documented how the genocide came about.} Needless to say, all of the subject populations of the above mass-murders had been disarmed beforehand.

The wars and mass-murders which the mortal governments routinely engage in are literal human-sacrifice orgies that the Earthly rulers of those governments offer up to appease their god, Satan!

Government, throughout all of recorded history, has been the most methodical and efficient human-meat grinder to ever exist. It is a purely Satanical machination masquerading as humanity's salvation, but has always been---and forever will be so long as it exists---the scourge of mankind and our decline.

\section{Jesus Defended the Right to Freely Contract and Private Property Rights}
\label{sec:JesusDefendedTheRightToFreelyContractAndPrivatePropertyRights}

Besides the Golden Rule which Jesus commanded as the ultimate social ethic, another Biblical account of Jesus's teachings which clearly demonstrates His attitude toward the institution of private property and the free and voluntary trade thereof is given in His below Parable of the Workers in the Vineyard:

\begin{squotation}
``For the kingdom of heaven is like a landowner who went out early in the morning to hire laborers for his vineyard. Now when he had agreed with the laborers for a denarius a day, he sent them into his vineyard. And he went out about the third hour and saw others standing idle in the marketplace, and said to them, `You also go into the vineyard, and whatever is right I will give you.' So they went. Again he went out about the sixth and the ninth hour, and did likewise. And about the eleventh hour he went out and found others standing idle, and said to them, `Why have you been standing here idle all day?' They said to him, `Because no one hired us.' He said to them, `You also go into the vineyard, and whatever is right you will receive.'

``So when evening had come, the owner of the vineyard said to his steward, `Call the laborers and give them their wages, beginning with the last to the first.' And when those came who were hired about the eleventh hour, they each received a denarius. But when the first came, they supposed that they would receive more; and they likewise received each a denarius. And when they had received it, they complained against the landowner, saying, `These last men have worked only one hour, and you made them equal to us who have borne the burden and the heat of the day.' But he answered one of them and said, `Friend, I am doing you no wrong. Did you not agree with me for a denarius? Take what is yours and go your way. I wish to give to this last man the same as to you. Is it not lawful for me to do what I wish with my own things? Or is your eye evil because I am good?' So the last will be first, and the first last. For many are called, but few chosen.''\footnote{Matthew 20:1--16.}
\end{squotation}

It never ceases to amaze me when socialists sometimes try to claim that Jesus was some sort of proto-socialist or Communist. Anyone who is the least bit familiar with the socialists' attitude toward such matters would know that the typical socialist response to such a landowner's actions towards his workers would be to scream bloody murder! Of course, a socialist government's response to such a landowner would be to exterminate him. Yet here Jesus reinforces the correctness of the libertarian creed on the absoluteness of lawfully being able to do what one wishes with their own possessions, as well as being able to freely and voluntarily contract said possessions as one sees fit---even if doing so greatly upsets others! So long as one has kept one's word in the contracts in which one has agreed to---and so long as one's actions pertain to their own property---then the right of the individual to make decisions concerning their property remains absolute!

\section{Greatness Is in Serving}
\label{sec:GreatnessIsInServing}

One of the things which most clearly demonstrates just how different Jesus's Kingdom is to be from the mortal, Earthly kingdoms and governments---and also why we should be very careful to never confuse the two together---is given in the story of when the apostles James and John came to Jesus asking if they may have the favor granted to them of being able to sit on either side of Jesus's throne, one to the right and the other to His left, and this is how Jesus answered them:

\begin{squote}
But Jesus called them to Himself and said to them, ``You know that those who are considered rulers over the Gentiles lord it over them, and their great ones exercise authority over them. Yet it shall not be so among you; but whoever desires to become great among you shall be your servant. And whoever of you desires to be first shall be slave of all. For even the Son of Man did not come to be served, but to serve, and to give His life a ransom for many.''\footnote{Mark 10:42--45. See also Matthew 18:4; 20:25--28; Mark 9:35; Luke 22:25,26.}
\end{squote}

How diametrically opposite the Kingdom of Christ is indeed from that of the mortal, Earthly governments! Thus, when it is claimed herein that Jesus is an ``anarchist'' it needs to be borne in mind that this is in relation to how all mortal governments on Earth have operated. If one wishes to refer to the ``Government of Christ'' or the ``Kingdom of Christ,'' this is fine so long is one realizes that the Government of Christ will in no sense be an actual government as they exist on Earth and are controlled by mortals.

It needs to also be pointed out here that above in Mark 10:42 Jesus rebukes the supposed ``authority'' of the Earthly ``rulers''! Thus He says of them ``You know that those who are considered rulers over the Gentiles lord it over them, and their great ones exercise authority over them''---here is clear proof that just because mortals on Earth may consider someone to be a ``ruler'' does not mean that God considers them to be a genuine ruler! In the eyes of God, those who are the greatest among men are those who seek to serve their fellow men, not those who seek to be served by their fellow men!

\section{Slaves Obey Your Masters?}
\label{sec:SlavesObeyYourMasters}

While although not directly related to the issue of the ethical status of government \emph{per se}, some individuals have asserted that certain statements in the New Testament by Paul and Peter condone the institution of slavery, and for this reason it is important as it concerns social relations in general. Thus, Paul writes:

\begin{squotation}
Bondservants, be obedient to those who are your masters according to the flesh, with fear and trembling, in sincerity of heart, as to Christ; not with eyeservice, as men-pleasers, but as bondservants of Christ, doing the will of God from the heart, with goodwill doing service, as to the Lord, and not to men, knowing that whatever good anyone does, he will receive the same from the Lord, whether he is a slave or free.

And you, masters, do the same things to them, giving up threatening, knowing that your own Master also is in heaven, and there is no partiality with Him.\footnote{Ephesians 6:5--9.}
\end{squotation}

\begin{squotation}
Bondservants, obey in all things your masters according to the flesh, not with eyeservice, as men-pleasers, but in sincerity of heart, fearing God. And whatever you do, do it heartily, as to the Lord and not to men, knowing that from the Lord you will receive the reward of the inheritance; for you serve the Lord Christ. But he who does wrong will be repaid for what he has done, and there is no partiality.

Masters, give your bondservants what is just and fair, knowing that you also have a Master in heaven.\footnote{Colossians 3:22--4:1.}
\end{squotation}

\begin{squote}
Let as many bondservants as are under the yoke count their own masters worthy of all honor, so that the name of God and His doctrine may not be blasphemed. And those who have believing masters, let them not despise them because they are brethren, but rather serve them because those who are benefited are believers and beloved. Teach and exhort these things.\footnote{1~Timothy 6:1,2.}
\end{squote}

\begin{squote}
Exhort bondservants to be obedient to their own masters, to be well pleasing in all things, not answering back, not pilfering, but showing all good fidelity, that they may adorn the doctrine of God our Savior in all things.\footnote{Titus 2:9,10.}
\end{squote}

And Peter writes:

\begin{squotation}
Servants, be submissive to your masters with all fear, not only to the good and gentle, but also to the harsh. For this is commendable, if because of conscience toward God one endures grief, suffering wrongfully. For what credit is it if, when you are beaten for your faults, you take it patiently? But when you do good and suffer, if you take it patiently, this is commendable before God. For to this you were called, because Christ also suffered for us, leaving us an example, that you should follow His steps:

\begin{sverse}
``Who committed no sin,\\
Nor was deceit found in His mouth'';
\end{sverse}

who, when He was reviled, did not revile in return; when He suffered, He did not threaten, but committed Himself to Him who judges righteously; who Himself bore our sins in His own body on the tree, that we, having died to sins, might live for righteousness---by whose stripes you were healed. For you were like sheep going astray, but have now returned to the Shepherd and Overseer of your souls.\footnote{1~Peter 2:18--25.}
\end{squotation}

But to quote the above passages as condoning the institution of slavery, one would thereby be confusing the offering of pragmatic advice on how to best handle a bad situation as granting the rightness of that situation. Yet obviously Peter and Paul didn't so regard the institution of slavery as being at all just, for then there would have been no cause for Peter compare the slave's suffering to that of Jesus in 1~Peter 2:21--25---as certainly any true Christian regards the scourging and execution of Jesus to have been unjust, to say the very least. Thus the fact that Peter \emph{did} compare the slave's suffering to that of Jesus is by itself enough to demonstrate that he considered it to be unjust.

So what of the actual ethical status of the institution of slavery as it concerns Jesus's own teachings? On this question there can be no doubt, as one of the main reasons Jesus came was to call us to liberty! Jesus said this Himself as recorded in Luke 4:16--21:

\begin{squotation}
So He came to Nazareth, where He had been brought up. And as His custom was, He went into the synagogue on the Sabbath day, and stood up to read. And He was handed the book of the prophet Isaiah. And when He had opened the book, He found the place where it was written:

\begin{sverse}
``The Spirit of the \textsc{Lord} is upon Me,\\
Because He has anointed Me\\
To preach the gospel to the poor;\\
He has sent Me to heal the brokenhearted,\\
To proclaim liberty to the captives\\
And recovery of sight to the blind,\\
To set at liberty those who are oppressed;\\
To proclaim the acceptable year of the \textsc{Lord}.''
\end{sverse}

Then He closed the book, and gave it back to the attendant and sat down. And the eyes of all who were in the synagogue were fixed on Him. And He began to say to them, ``Today this Scripture is fulfilled in your hearing.''
\end{squotation}

So here we have it: Jesus Himself said that He came to proclaim liberty to the captives and to set at liberty the oppressed!

The word ``liberty'' in Luke 4:18 is a translation of the Greek word \emph{aphesei} (\ibygr{afesei}), and means: release from bondage or imprisonment; forgiveness or pardon, i.e., remission of the penalty. Thus, it is a complete and absolute negation of the condition of being a slave. And since there exists no recorded instance of Jesus qualifying the above statement, it then becomes quite clear that Jesus is very much against the institution of slavery---besides of course slavery being totally incompatible with the Golden Rule.

So how are we to make better sense of Paul and Peter's above statements, since it is clear that the institution of slavery is very anti-Christian in the most literal sense of the word (i.e., as it concerns the actual doctrine as preached by Jesus Christ)?

One must bear in mind that Paul and Peter knew better than most of the injustices contained within this Satanic world system: Paul himself was continuously in and out of prisons during his ministry, and would at last be beheaded by government for preaching the gospel of Christ, just as John the Baptist was beheaded by government before him for preaching the same. In 1~Corinthians 9:19--23 Paul clarifies his above statements while at the same time declaring the absoluteness of his God-given rightful liberty:

\begin{squote}
For though I am free from all men, I have made myself a servant to all, that I might win the more; and to the Jews I became as a Jew, that I might win Jews; to those who are under the law, as under the law, that I might win those who are under the law; to those who are without law, as without law (not being without law toward God, but under law toward Christ), that I might win those who are without law; to the weak I became as weak, that I might win the weak. I have become all things to all men, that I might by all means save some. Now this I do for the gospel's sake, that I may be partaker of it with you.
\end{squote}

It is here where the seeming contradiction of certain passages in the Bible whereby Paul and Peter admonish slaves to ``obey their masters''\footnote{See Ephesians 6:5; Colossians 3:22; 1~Timothy 6:1; Titus 2:9; 1~Peter 2:18.} is cleared up. Such an admonition is a pragmatic one, not a categorical moral one: as Paul himself declared his absolute rightful freedom from all men, and also called for people to ``Imitate me, just as I also imitate Christ''!\footnote{1~Corinthians 11:1.} So rather than using defensive force to repel one's Earthly ``master,'' or trying to run away---which in the end would probably only affect one's freedom in a negative way---a much better and effective solution would be to convert one's Earthly ``master'' to Jesus, and if one has truly succeeded in doing so, whereby one's Earthly ``master'' becomes indwelt by the Holy Spirit, then one will have at once gained one's God-given absolute liberty, at least in relation to what the positive law considers one's ``master.'' The reason that this is necessarily the case is because Jesus commanded the absolute law as treating others as you would want others to treat you,\footnote{See Matthew 7:12; Luke 6:31.} yet the very institution of slavery is founded upon the exact opposite principle, as Abraham Lincoln pointed out (if only it had been that Lincoln himself had bothered to follow the logic of his below argument!):

\begin{squote}
If A can prove, however conclusively, that he may of right enslave B, why may not B snatch the same argument and prove equally that he may enslave A? You say A is white and B is black. It is color, then; the lighter having the right to enslave the darker? Take care. By this rule you are to be slave to the first man you meet with a fairer skin than your own. You do not mean color exactly? You mean the whites are intellectually the superiors of the blacks, and therefore have the right to enslave them? Take care again. By this rule you are to be slave to the first man you meet with an intellect superior to your own. But, say you, it is a question of interest, and if you make it your interest you have the right to enslave another. Very well. And if he can make it his interest he has the right to enslave you.\footnote{Abraham Lincoln, ``Fragment. On Slavery [July 1, 1854?]'' in John G. Nicolay and John Hay (editors), \emph{Complete Works of Abraham Lincoln, Volume II} (New York: Francis D. Tandy Company, New and Enlarged Edition, 1905; orig. ed., 1894), p. 186.}
\end{squote}

In the above discussion on the Golden Rule as commanded by Jesus it was pointed out that to not follow the Golden Rule is to do the opposite of the Golden Rule: i.e., to treat others as you would \emph{not} want others to treat you---of which ethic was termed the Satanic Principle.\footnote{See the discussion on this in Section \ref{sec:TheGoldenRuleUnavoidablyResultsInAnarchism} on page \pageref{SatanicPrinciple} of this article as to why such a designation logically follows.} Yet this is the very principle on which the institution of slavery necessarily rests.

And in further support of my contention that the conversion of a slave's Earthly ``master'' to Jesus would be the most effective and practical solution in obtaining one's God-given absolute liberty---at least in connection to what the positive law considers one's ``master''---consider Paul's own words on this matter from 2~Corinthians 3:17: ``Now the Lord is the Spirit; and where the Spirit of the Lord is, there is liberty.''

The word ``liberty'' in 2~Corinthians 3:17 is a translation of the Greek noun \emph{eleutheria} (\ibygr{e)leuqeri'a}) and is completely congruent in meaning with the English word ``liberty,'' i.e., \label{EleutheriaDefinition}as in freedom from slavery, independence, absence of external restraint, a negation of control or domination, freedom of access, etc. Thus, it is the complete negation of a state of slavery. But in fact, Paul even goes further than this in the very passages above which some have contended condone the institution of slavery. Thus, in Ephesians 6:9 Paul writes:

\begin{squote}
And you, masters, do the same things to them, giving up threatening, knowing that your own Master also is in heaven, and there is no partiality with Him.
\end{squote}

Yet it is plainly clear that if a slave's ``master'' were to actually and truly give up \emph{threatening}---\emph{of all things}---then there can hardly even be said to exist a state of slavery any more in relation to what the positive law considers the ``master'' and the ``slave,'' as the very institution of slavery is enforced by the threat of either physical harm for noncompliance or recapture in the case of escape. Thus, this passage is actually a case of advocating the \emph{de facto} abolition of slavery even while a state of \emph{de jure} slavery---as considered by the positive law---may still be in place!

It is for the above reasons why the above-cited passages which some have contended condone the institution of slavery can only make any sense within the Christian point of view as pragmatic advice on how best to handle a bad and unjust situation, and certainly cannot be regarded as commentary on the ethical rightfulness of the institution of slavery; nor for that matter as a categorical moral imperative as to how one is always to conduct oneself---as Paul and Peter were often in rebellion against what the positive law considered their ``masters.'' Extreme problems arise for those who would try to contend otherwise---for just one example of the problems presented to those who would thus contend, consider Paul writing in 1~Timothy 5:23 to ``No longer drink only water, but use a little wine for your stomach's sake and your frequent infirmities.''

Yet this statement by Paul is completely unqualified, and far more direct than his above advice to slaves. Thus, for those who would contend that Paul was giving a categorical moral imperative as to how a slave is always to conduct himself in relation to his ``master''---as opposed to merely offering advice as to the best and most practical way in which a slave could go about obtaining his God-given liberty in relation to his ``master''---such individuals, if they are to be consistent, would also have to contend that according to Paul it is a sin not to drink wine for one's stomach's sake! In fact the case for this contention would actually be much stronger than in that of Paul's advice to slaves, for unlike in his advice to slaves nowhere does Paul qualify the above statement! Yet obviously to make such a contention would be absurd, as in both cases it would be confusing pragmatic advice with a categorical moral imperative.

But moreover, here is what Jesus Himself had to say about the serving of masters:

\begin{squote}
``No one can serve two masters; for either he will hate the one and love the other, or else he will be loyal to the one and despise the other. You cannot serve God and mammon.''\footnote{Matthew 6:24. See also Luke 16:13.}
\end{squote}

Yet what in the world is the institution of slavery if not mammon? If the institution of slavery does not qualify as mammon then there is nothing that possibly could! For it is a method of obtaining wealth that is a complete and utter violation of Jesus's ultimate ethical commandment: ``Therefore, whatever you want men to do to you, do also to them, for this is the Law and the Prophets.''\footnote{Matthew 7:12. See also Luke 6:31.}

Thus it becomes clear that the institution of slavery is just another product of this sick Satanic world system---of which system Jesus is to ultimately overthrow in the time of His Judgement. Mammon indeed!

\section{Jesus on the Collection of Interest (i.e., Usury)}
\label{sec:JesusOnTheCollectionOfInterestIEUsury}

One of the socialists' great bugbears has been the institution of usury, or otherwise the collecting of interest. Yet in the only instance in which Jesus commented upon this He was clearly in favor of the concept, as is given in His Parable of the Talents, in which a man traveling to a far-away country leaves his three servants with some talents to make use of in the best way they see fit while he is away---the first two servants invest the talents and receive more talents from their initial investment, and this makes the lord of the estate happy to hear this upon his return; but here is what Jesus says of the third servant:

\begin{squotation}
``Then he who had received the one talent came and said, `Lord, I knew you to be a hard man, reaping where you have not sown, and gathering where you have not scattered seed. And I was afraid, and went and hid your talent in the ground. Look, there you have what is yours.'

``But his lord answered and said to him, `You wicked and lazy servant, you knew that I reap where I have not sown, and gather where I have not scattered seed. So you ought to have deposited my money with the bankers, and at my coming I would have received back my own with interest.'\thinspace''\footnote{Matthew 25:24--27. See also Luke 19:21--23.}
\end{squotation}

Now obviously this parable is a lesson on how Christians should be ever-vigilant in converting people to salvation in Christ, in that we should not keep the Gospel of Christ to ourselves but always seek to increase the number of Christians in the world. But even so, it nevertheless demonstrates that Jesus was hardly hostile to the concept of the collecting of interest, considering that this was his only commentary given on the institution of interest.

It might be pointed out that in Luke 6:34,35, Jesus tells us to lend without even expecting to receive \emph{anything} back! Jesus spoke this in the wider context recorded in Luke 6:27--38, which has to do with the forgiveness of wrongs committed against us, including debts owed to us.\footnote{See also Matthew 5:40--42.} This forgiveness is something that we as Christians should want to do out of our own voluntary free-will, and hence should not be conflated with the use of force involved in government laws against people making mutual agreements.

Moreover, the above view ties in quite appropriately with Jesus's attitude toward the absolute lawfulness of an individual doing what they wish with their own property---including freely contracting thereof---as told by Jesus in his Parable of the Workers in the Vineyard as recorded in Matthew 20:1--16.\footnote{See Section \ref{sec:JesusDefendedTheRightToFreelyContractAndPrivatePropertyRights} on page \pageref{sec:JesusDefendedTheRightToFreelyContractAndPrivatePropertyRights} of this article.}

\section{The Cleansing of the Temple}
\label{sec:TheCleansingOfTheTemple}

The only recorded tumultuous outburst by Jesus was what is now known as the Cleansing of the Temple:

\begin{squote}
Then Jesus went into the temple of God and drove out all those who bought and sold in the temple, and overturned the tables of the money changers and the seats of those who sold doves. And He said to them, ``It is written, `My house shall be called a house of prayer,' but you have made it a `den of thieves.'\thinspace''\footnote{Matthew 21:12,13. See also Mark 11:15--17; Luke 19:45,46; John 2:14--17. See Nehemiah 13:7--10 for a previous prophetic temple-cleansing.}
\end{squote}

Now this event is often misinterpreted as being some sort of revolt by Jesus on the bad aesthetics of commerce being conducted inside of God's temple, and so is sometimes given as anti-libertarian and anti-free-market commentary. But if that were really what this episode was about then there would have been no cause for Jesus to accuse the priests of turning the temple into a ``den of thieves.''

Jesus was being literal when He said that. To understand what Jesus was talking about one has to understand the nature of what was being bought and sold in the temple as well as the function of the ``money changers.'' What was being bought and sold in the temple were animals which were to be sacrificed as a sin offering, and the function of the money changers was to convert the Gentile Roman money into the Jewish money which would then be suitable to present inside the temple for purchase of the sacrificial animals. The people who bought these animals did not get to take them home to eat---if they had then Jesus would have had no good reason to object to the commerce being conducted at the temple, and certainly would have no grounds to accuse the priests of thievery. Rather, the animals stayed in the temple to be sacrificed by the Levitical priests, which by so doing would (as it was supposed) atone for the sins of the purchaser of the sacrificed animal. So when Jesus accused the priests who conducted this practice of being thieves, what He was saying was that the people who bought these animals to be sacrificed to atone for their sins were being \emph{ripped off}---i.e., that the animal sacrifices weren't doing anything for their sins. In other words, the priests were selling religious snake-oil: misrepresenting their product as curing something it couldn't cure; hence they were committing fraud (per libertarian rights theory).

Now realize what is at stake here: Jesus came to save people's very souls, and here people are being deceived and defrauded into believing that sacrificing these animals is setting their souls right with God. As it is written in Hebrews 10:4--7:

\begin{squotation}
For it is not possible that the blood of bulls and goats could take away sins. Therefore, when He came into the world, He said:

\begin{sverse}
``Sacrifice and offering You did not desire,\\
But a body You have prepared for Me.\\
In burnt offerings and sacrifices for sin You had no pleasure.\\
Then I said, `Behold, I have come---\\
In the volume of the book it is written of Me---\\
To do Your will, O God.'\thinspace''
\end{sverse}

Previously saying, ``Sacrifice and offering, burnt offerings, and offerings for sin You did not desire, nor had pleasure in them'' (which are offered according to the \mbox{law) \ldots}\footnote{See also Psalms 40:6--8; Isaiah 1:11--14; Jeremiah 7:21,22; 8:8; Hosea 6:6; Amos 5:21,22.}
\end{squotation}

If we consider that Jesus is God's Messiah, then He was in a particularly unique position to accurately determine whether or not these animal sacrifices were achieving what was being claimed for them, and it was determined by Jesus that the priests were defrauding their patrons. Thus, per libertarianism (Rothbardian theory in particular), He took appropriate action by using retaliatory force against these thieves. It is important to point out that it is only a true prophet from God who could have taken such action, for any normal man would not have possessed the requisite information in order to make that determination honestly. In this manner, not only was Jesus's only use of force quite libertarian, but it was also in a situation which would have been inappropriate for most anyone else.

The foregoing paragraph concerns cleansing the temple based upon prophetic knowledge. However, the temple and its furnishings were dedicated to God, with the temple being regarded as the house of God,\footnote{See Ezra 6:17.} and so God in His Sonship aspect (see Isaiah 9:6) was simply overturning His own tables inside His own temple without laying a hand on anyone. Understood in this light, Jesus wasn't even using retributive force against the temple property or anyone's person.

\section{Jesus on the War on Drugs (and All Forms of Prohibition)}
\label{sec:JesusOnTheWarOnDrugsAndAllFormsOfProhibition}

In the modern era one of the most virulent scourges which has plagued the Western societies in particular is the so-called ``drug problem,'' i.e., the use of, and combating the use of, illegal drugs. Yet, why has the ``drug problem'' only become such a problem within, predominately, the last century? What is the cause of this? But first, before we answer this question, the more important issue from the Christian's viewpoint is: what is Jesus's position on the so-called ``drug problem,'' i.e., whether it is called the ``War on Drugs'' or ``Prohibition''? More directly, what does Jesus have to say about prohibiting by law the use of certain drugs, or inebriants?

A number of people may be thinking that the issue only concerns \emph{which} drugs or inebriants ought to be prohibited and how severe the penalty for their use should be---as those calling themselves Christians have traditionally been at the forefront of not only the Prohibition of alcohol during the '20s in the U.S., but so also with the continuing War on Drugs. So, first of all, what does Jesus have to say about which substances ought to be outlawed?

On this question Jesus is quite clear about it in no uncertain terms---although the answer may come as a surprise to some: absolutely no law ought to exist prohibiting the consumption of any substance whatsoever! Jesus says quite clearly in the strongest of terms that there is no substance a man can consume that could possibly defile him---thus we read in Mark 7:15--23:

\begin{squotation}
``There is nothing that enters a man from outside which can defile him; but the things which come out of him, those are the things that defile a man. If anyone has ears to hear, let him hear!''

When He had entered a house away from the crowd, His disciples asked Him concerning the parable. So He said to them, ``Are you thus without understanding also? Do you not perceive that whatever enters a man from outside cannot defile him, because it does not enter his heart but his stomach, and is eliminated, thus purifying all foods?'' And He said, ``What comes out of a man, that defiles a man. For from within, out of the heart of men, proceed evil thoughts, adulteries, fornications, murders, thefts, covetousness, wickedness, deceit, lewdness, an evil eye, blasphemy, pride, foolishness. All these evil things come from within and defile a man.''\footnote{See also Matthew 15:11,17--20.}
\end{squotation}

This is the only directive that Jesus gives in the entire Bible as to what substances should be, or should not be, prohibited. Some may claim that Jesus was only talking about food in the above, and not psychotropic drugs. Yet if this were truly the case then Jesus's above claim is a false one: Jesus saying ``There is nothing that enters a man from outside which can defile him'' would be wrong, for then there would indeed be something which could thereby defile a man---namely: psychotropic drugs! Yet Jesus is absolutely clear on this issue: there is no substance a person can consume which could possibly defile them!

Also, just about any ``drug,'' in principle, can also be made into a ``food''---and traditionally often have been and continue to be: thus, the drug ethanol is almost always consumed not by itself, but in combination with non-inebriants as a drink; the drug caffeine is almost always consumed as the beverage known as coffee; marijuana has often been consumed as an edible baked into brownies; cocaine was once an ingredient in the original formulation of the name-brand soft-drink Coca-Cola; etc. If the modern-day Prohibitionists desire to maintain that Jesus did not mean to include substances such as psychotropic drugs when He gave this clear directive then the burden is on them to show where in the Bible Jesus qualifies His above statement to include the possibility that psychotropic drugs are an exception to His above all-inclusive directive. But search the Bible high and low and no such alternate, qualifying directive is anywhere to be found.

Some may be quick to point out that John writes in Revelation 9:21, ``And they did not repent of their murders or their sorceries or their sexual immorality or their thefts''\footnote{See also Revelation 18:23; 21:8; 22:15.} and that the word that is here translated as ``sorceries'' is in the original Greek \emph{pharmakeion} (\ibygr{farmakeion}), which is the genitive form of \emph{pharmakeia} (\ibygr{farmakei'a}), i.e., as in ``pharmaceutics'' or ``drugs.'' But the original sense of this Greek word \emph{pharmakeia} meant the mixing of various ingredients for magical purposes, whether or not they were actually ever intended to be consumed by anyone, or whether or not they had what we would call today ``pharmacological'' properties: in other words, it was for the most part pure spell-casting---often black-magic in nature, such as casting hexes on people. Thus, the most accurate translation of this word into modern English is indeed ``sorceries,'' and not ``drugs''---and this is indeed how almost all English Bible translations have handled this word: whether it be the King James Version or almost all modern translations. But even if such were not the case and one were to maintain that \emph{pharmakeia} here really did mean ``drugs'' then this would present such a person with quite a serious problem: \emph{which drugs?} If indeed one were to maintain that \emph{pharmakeia} here should be translated as ``drugs'' then one would logically have to so also maintain that \emph{all} drugs are thereby meant by it, regardless of whatever psychotropic properties they may or may not have---the reason being is because no type of drug in particular would then be specified in the above Bible passages. Thus, there would then be no grounds for singling out psychotropic drugs such as ethanol over, say, penicillin, or any other life-preserving medicine for that matter. To be consistent, some may get around this problem by saying: very well, \emph{all drugs}, including medicine, are thereby meant by it. But to so maintain this would just create an even bigger problem than the one it just solved: for the Bible teaches that ``A merry heart does good, like medicine, but a broken spirit dries the bones''\footnote{Proverbs 17:22.}; and the angel sent to Ezekiel, in the description of the Heaven on Earth that Jesus is to establish after the Judgement, said of it, in part:

\begin{squote}
``Along the bank of the river, on this side and that, will grow all kinds of trees used for food; their leaves will not wither, and their fruit will not fail. They will bear fruit every month, because their water flows from the sanctuary. Their fruit will be for food, and their leaves for medicine.''\footnote{Ezekiel 47:12.}
\end{squote}

So quite simply put, if one were to so maintain that all drugs must be meant by the above passages in Revelation then one would be going against Biblical doctrine, as what little the Bible does have to say about medicinal drugs it is nevertheless clear about: that curative drugs are a good thing. Thus, if these passages in Revelation actually meant ``drugs'' instead of ``sorceries'' then the Bible would be contradicting itself here, as the passages in Revelation would thereby be inclusive of all drugs, not just any kind in particular. But even if we were to here grant for argument's sake that one could somehow narrow it down to some sort of drug types in particular, one still would not be able to derive that such drugs should therefore be outlawed, as nowhere would these passages in Revelation then so much as even suggest that mortal governments make any laws against such drugs.

Thus, even under the most favorable interpretation of the Bible---from the viewpoint of modern-day Prohibitionists---Jesus's declaration that ``There is nothing that enters a man from outside which can defile him'' would still stand---at least as it concerned all mortal, Earthly forms of judgement.

Some diligent readers may now say at this point, to the effect of: ``Wait a minute! The Mark of the Beast is an obvious exception, as this is something which possibly enters a man from the outside which defiles him!''\footnote{The King James Version translates the Mark of the Beast in Revelation 13:16 as being ``in'' the hand or forehead, while most modern Bible versions translate it as being ``on.'' Besides being more accurate to the Greek word \emph{epi} (\ibygr{epi}), another advantage of translating the Mark as being ``on'' the hand or forehead is that this then, in almost all cases, covers both possibilities: as in almost all cases, in order to put some identifying mark ``in'' the skin would require that one also leave a mark ``on'' the skin.} But this would ignore Jesus's follow-up elaboration about all such substances under discussion eventually being ``eliminated'' from the body by its natural excretion processes, as the Mark of the Beast is meant to be a lifelong identifier, and thus is not excreted by the body's natural processes, as are eventually all foods and drugs. But if one still wants to persist in this line of reasoning they may counter that indeed not all drugs are eliminated by the body's natural excretion processes: of those who die of drug overdoses, the drugs which thereby caused their deaths are not then excreted by the body's natural processes. While although this is quite true, one would still not be able to derive therefore from it that there ought to be laws against certain drugs, as all drugs are capable of causing death from overdose; indeed, most lethal drug overdoses are not caused by illegal psychotropic drugs, but legally used medicines---and hence, one would be presented with the original problem discussed above on this. And, it should be stated in passing, it would also be completely nonsensical to make a law against taking a lethal overdose of a drug, as the penalty for taking a lethal dose of drugs would be, by definition, an automatic death-penalty: therefore any such lawbreaker would automatically be out of the reach of any Earthbound, mortal law-enforcer.

Thus, any which way one slices it, it is simply quite impossible to justify any form of drug-prohibition whatsoever from a Biblical perspective. But even far stronger than such drug-laws being merely unjustifiable from a Biblical perspective, all such laws go directly against Jesus's clear directive that all things which a person may consume cannot possibly defile them! And thus, not only are all drug-laws extra-Biblical in origin, they are all also extremely anti-Christian in the most literal sense of the word! If there should be the slightest shred of doubt left in one's mind as to the veracity of this, then hereby, once and for all, let Paul slay that misplaced sense of doubt:

\begin{squote}
Therefore, if you died with Christ from the basic principles of the world, why, as though living in the world, do you subject yourselves to regulations---``Do not touch, do not taste, do not handle,'' which all concern things which perish with the using---according to the commandments and doctrines of men? These things indeed have an appearance of wisdom in self-imposed religion, false humility, and neglect of the body, but are of no value against the indulgence of the flesh.\footnote{Colossians 2:20--23. See also Romans 14:14.}
\end{squote}

So we see in no uncertain terms that all forms of drug-prohibition are completely unjustifiable from a Biblical viewpoint, and indeed anti-Christian. If then such drug-laws are extra-Biblical and anti-Christian, how is it that many self-professed Christians came to be on the forefront of all the various forms of drug-prohibition within recent history? Quite amazingly, this very question was already answered almost 2000 years ago by Paul, and in shocking but no uncertain terms:

\begin{squote}
Now the Spirit expressly says that in latter times some will depart from the faith, giving heed to deceiving spirits and doctrines of demons, speaking lies in hypocrisy, having their own conscience seared with a hot iron, forbidding to marry, and commanding to abstain from foods which God created to be received with thanksgiving by those who believe and know the truth. For every creature of God is good, and nothing is to be refused if it is received with thanksgiving; for it is sanctified by the word of God and prayer.\footnote{1~Timothy 4:1--5.}
\end{squote}

As was already pointed out above, just about any ``drug,'' in principle, can also be made into a ``food''---and traditionally often have been and continue to be. Indeed, the first truly large-scale form of drug-prohibition in a Western society in the modern era was what was known as Prohibition in the U.S., which was the outlawing of consuming the drug ethanol, i.e., ``alcohol.'' Yet alcohol is consumed almost exclusively as a food-stuff in mixture with non-inebriating potables! Indeed, straight laboratory-grade ethanol is virtually inedible, if not actually quite painful to so consume. So how very true and accurate Paul was when he wrote the above words, as it was predominately self-professed Christians who lead the movement to outlaw the food of alcoholic beverages! And to grasp the awful extent that these self-professed Christians must have been truly deceived by demons in order to prohibit the food of alcoholic beverages, just consider that the first miracle recorded in the Bible by Jesus was to turn water into wine during the wedding at Cana!\footnote{See John 2:9--11.} What absolute blasphemy for them to prohibit the resultant product of the first miracle of their proclaimed God! Deceived by demons indeed! Truer words could not have been written by Paul to describe such a perverted situation.

Indeed, it was Paul himself that counseled us to ``No longer drink only water, but use a little wine for your stomach's sake and your frequent infirmities.''\footnote{1~Timothy 5:23.} And Psalms 104:14,15 says of God that

\begin{sverse}
He causes the grass to grow for the cattle,\\
And vegetation for the service of man,\\
That he may bring forth food from the earth,\\
And wine that makes glad the heart of man,\\
Oil to make his face shine,\\
And bread which strengthens man's heart.\footnote{See also Judges 9:13.}
\end{sverse}

Many in the Temperance movement responsible for Prohibition had falsely claimed that these Biblical references to ``wine'' were in reality grape juice. But the Greek word for wine in the New Testament, \emph{oinos}, is a fermented drink, whereas the Greek word for fruit juice is \emph{khymos}. And besides that, this claim demonstrates either an appalling ignorance of Jesus's own parables or outright deceit, as Jesus even referred to the fermenting of wine in one of his parables:

\begin{squote}
``No one puts a piece of unshrunk cloth on an old garment; for the patch pulls away from the garment, and the tear is made worse. Nor do they put new wine into old wineskins, or else the wineskins break, the wine is spilled, and the wineskins are ruined. But they put new wine into new wineskins, and both are preserved.''\footnote{Matthew 9:16,17. See also Mark 2:22; Luke 5:37.}
\end{squote}

In ancient times goatskins were used to hold wine. As the fresh grape juice fermented, carbon dioxide would be generated by the living yeast's metabolism, increasing the volume of gas contained in the wineskin, and so the new wineskin would stretch. But a used wineskin, already stretched, would break. Not only that, but before 1869 it was impossible to store grape juice in temperate to hot climates (which are the climates grapes grow in) without it either quickly going bad or becoming wine. If grape juice is left exposed to the open air then it will quickly go bad due to mold and bacteria---sealing grape juice from the open air protects it from these aerobic microorganisms because the yeast which is present naturally on the grapes creates an atmosphere of carbon dioxide while at the same time making alcohol. Consequently, storing non-alcoholic grape juice was an impossibility until 1869, when Dr. Thomas Bramwell Welch succeeded in applying the process of pasteurization to freshly squeezed must. About the only people who may have been drinking grape juice before 1869 were those who pressed the freshly picked grapes themselves (without refrigeration grapes will quickly go bad, unless they are dried into raisins). It is for this reason that the suggestion that the fruit of the vine that Jesus and the twelve disciples drank during the Last Supper on Passover was grape juice is absurd,\footnote{See Mark 14:23--25.} as the growing season for grapes in Palestine is from April to October (the dry season), yet Passover starts on the 14th of the Jewish month Nisan (the actual Last Supper occurred on the 14th of Nisan), which is a lunar month that roughly corresponds with the latter part of March and the first part of April---so quite simply, there would have existed no unfermented grape juice at this time, as no grapes would have existed, since the growing season for them had just started.

In the beginning of this discussion on drugs, it was first inquired as to why the ``drug problem'' has only become such a problem within, predominately, the last century. To answer this: the reason is precisely because of the very laws against drugs! The government's War on Drugs has turned what once was an individual problem into a social problem by inventing new make-believe ``crimes'' that aggress against no one, while spawning a whole true crime industry associated with it (just like during Prohibition). The effect of libertarian legalization would be to make drugs an individual problem again instead of the grave social problem that it is today. As they say, we don't have a drug problem, we have a drug-problem problem. Were it not for the government's War on Drugs, the gang turf-wars, theft, and other various true crimes that are associated with the distribution of drugs and the procurement of money in which to support habituations to drugs---of which the price has been artificially inflated---would not exist.

How many liquor stores have shootouts between each other? Yet when alcohol was illegal, the black-market distributors of alcohol found it necessary to have shootouts and murders between each other on a regular basis. This was because, being that their business was illegal, they did not have access to the courts in which to settle their disputes; as well, because their business was illegal, this raised the stakes of doing business, for if they got caught then they would go to prison---thus it became profitable to resort to murder in order to solve problems which would otherwise lead to prison. And how many tobacco smokers resort to theft and prostitution in order to support their habit? Yet studies have shown that tobacco is more habit forming than heroin.\footnote{James C. Anthony, Lynn A. Warner and Ronald C. Kessler, ``Comparative Epidemiology of Dependence on Tobacco, Alcohol, Controlled Substances, and Inhalants: Basic Findings From the National Comorbidity Survey,'' \emph{Experimental and Clinical Psychopharmacology}, Vol. 2, No. 3 (August 1994), pp. 244--268.} The reason you don't see tobacco smokers doing such things is because tobacco addicts can afford to support their habit. When the Soviet Union experienced an artificial shortage of cigarettes in the summer of 1990 due to the collapse of its socialist economy, tobacco smokers took to the streets \emph{en masse} rioting, eventually requiring emergency shipments of Marlboros and other cigarette brands from the U.S. in order for it to cease.\footnote{James Rupert and Glenn Frankel, ``In Ex-Soviet Markets, U.S. Brands Took On Role of Capitalist Liberator,'' \emph{Washington Post}, November 19, 1996, p. A01.} If heroin or crack were legal it would cost no more (and probably less) than a tobacco habit, and so heroin and crack addicts would be able to support their habit by working at a regular job instead of resorting to theft and prostitution. If one should doubt this last statement, it should be borne in mind that the original laws in the U.S. against the use of opium were to punish the Chinese opium-smoking immigrants in the early 1900s, who were so productive that they were taking railway construction jobs away from white Americans.

As a parting note on this subject, I will leave you with what Peter counseled us: ``But let none of you suffer as a murderer, a thief, an evildoer, or as a busybody in other people's matters.''\footnote{1~Peter 4:15.} How very much this last admonition applies to all forms of drug-prohibition!

\section{Woe to Lawyers!}
\label{sec:WoeToLawyers}

In Jesus's day, as well as in modern times, lawyers have had quite a system worked out for themselves. Not only are lawyers the ones who write the laws, but they are also the ones who become rich in prosecuting and defending people from those very laws that they or their colleagues have written in the first place. As well, most politicians, especially in modern times, are also lawyers. Thus, throughout history there has existed a grotesque system whereby the very people responsible for the laws have a perverse incentive in making sure that they are as arcane, unintelligible, byzantine and numerous as possible---hence, always insuring a healthy demand for their services.

This fact was certainly not lost on Jesus, and He made a point to warn lawyers that they are putting their very souls at stake in their chosen profession. Thus, Jesus said:

\begin{squote}
``Woe to you, teachers of the law and Pharisees, you hypocrites! You shut the kingdom of heaven in men's faces. You yourselves do not enter, nor will you let those enter who are trying to.''\footnote{Matthew 23:13, New International Version.}
\end{squote}

Furthermore,

\begin{squotation}
And He said, ``Woe to you also, lawyers! For you load men with burdens hard to bear, and you yourselves do not touch the burdens with one of your fingers\mbox{.\hspace{0.167em}\ldots}

``Woe to you lawyers! For you have taken away the key of knowledge. You did not enter in yourselves, and those who were entering in you hindered.''\footnote{Luke 11:46,52.}
\end{squotation}

This is not to say that all lawyers throughout history are unrighteous. There has existed and does exist a few principled lawyers who entered their profession in order to defend righteous people from the unjust laws that their colleagues are responsible for---but they are and have been quite a minority indeed. The simple fact of the matter is that most lawyers are simply in it for the money, and generally have shown little to no interest in rolling back or defending against unjust laws if doing so negatively affects their bottom line. Even the ones that sometimes appear on the surface to be fighting against bad laws are often being paid quite handsomely in doing so, or are loyal opposition and have already been bought and paid for to purposely lose the case in order to, e.g., generate bad legal precedent in the case law, etc.

So a ``Christian lawyer'' is not an absolute contradiction in terms, it's just rather rare---and to the extent that such rare individuals do exist God has undoubtedly blessed them for their work in protecting His children against this Satanic world system. But in the main, how true indeed Jesus was being when He warned lawyers that they were jeopardizing their very souls in practicing the profession that they have chosen! Woe to lawyers indeed!

\section{Jesus on Government Courts: Avoid Them!}
\label{sec:JesusOnGovernmentCourtsAvoidThem}

Another thing which is quite congruent with Jesus's above warning to lawyers is Jesus's advice for the faithful to avoid the government's courts if at all possible:

\begin{squote}
``Agree with your adversary quickly, while you are on the way with him, lest your adversary deliver you to the judge, the judge hand you over to the officer, and you be thrown into prison. Assuredly, I say to you, you will by no means get out of there till you have paid the last penny.''\footnote{Matthew 5:25,26.}
\end{squote}

Also,

\begin{squote}
``Yes, and why, even of yourselves, do you not judge what is right? When you go with your adversary to the magistrate, make every effort along the way to settle with him, lest he drag you to the judge, the judge deliver you to the officer, and the officer throw you into prison. I tell you, you shall not depart from there till you have paid the very last mite.''\footnote{Luke 12:57--59.}
\end{squote}

Needless to say, government judges are also lawyers, so Jesus's advice here fits in with His warning to lawyers. It also completely demolishes the notion that Jesus considers what the government's positive law regards as ``authorities'' to be \emph{true} authorities---or otherwise Jesus would have no problem with such government judges resolving disputes among the faithful. In fact, Paul absolutely confirms this notion in 1~Corinthians 6:1--8:

\begin{squotation}
Dare any of you, having a matter against another, go to law before the unrighteous, and not before the saints? Do you not know that the saints will judge the world? And if the world will be judged by you, are you unworthy to judge the smallest matters? Do you not know that we shall judge angels? How much more, things that pertain to this life? If then you have judgments concerning things pertaining to this life, do you appoint those who are least esteemed by the church to judge? I say this to your shame. Is it so, that there is not a wise man among you, not even one, who will be able to judge between his brethren? But brother goes to law against brother, and that before unbelievers!

Now therefore, it is already an utter failure for you that you go to law against one another. Why do you not rather accept wrong? Why do you not rather let yourselves be cheated? No, you yourselves do wrong and cheat, and you do these things to your brethren!
\end{squotation}

And this also conclusively demonstrates that the ``authorities'' that Paul spoke of in Romans 13 could not possibly have been the ``authorities'' as so regarded by the government---as Paul said that the government judges ``are least esteemed by the church to judge''! Thus it is clear that he considered them to be no authority at all!

And so also James writes in James 2:6, ``But you have dishonored the poor man. Do not the rich oppress you and drag you into the courts?''

It needs to be pointed out that most of the rich in the days in which the above passage was written were rich due to grants of privilege by the government---particularly that of collecting taxes. Thus when James writes in the above of the rich oppressing the faithful and dragging them into the courts, he is speaking of actual violations of individuals' just property rights, and not of individuals reneging on voluntary contracts in which they had entered into. And this brings us naturally to the next point which needs to be made.

\section{Jesus on the Rich}
\label{sec:JesusOnTheRich}

Jesus had this to say about the rich:

\begin{squotation}
Now a certain ruler asked Him, saying, ``Good Teacher, what shall I do to inherit eternal life?''

So Jesus said to him, ``Why do you call Me good? No one is good but One, that is, God. You know the commandments: `Do not commit adultery,' `Do not murder,' `Do not steal,' `Do not bear false witness,' `Honor your father and your mother.'\thinspace''

And he said, ``All these things I have kept from my youth.''

So when Jesus heard these things, He said to him, ``You still lack one thing. Sell all that you have and distribute to the poor, and you will have treasure in heaven; and come, follow Me.''

But when he heard this, he became very sorrowful, for he was very rich.

And when Jesus saw that he became very sorrowful, He said, ``How hard it is for those who have riches to enter the kingdom of God! For it is easier for a camel to go through the eye of a needle than for a rich man to enter the kingdom of God.''

And those who heard it said, ``Who then can be saved?''

But He said, ``The things which are impossible with men are possible with God.''

Then Peter said, ``See, we have left all and followed You.''

So He said to them, ``Assuredly, I say to you, there is no one who has left house or parents or brothers or wife or children, for the sake of the kingdom of God, who shall not receive many times more in this present time, and in the age to come eternal life.''\footnote{Luke 18:18--30. See also Matthew 19:16--30; Mark 10:17--31.}
\end{squotation}

Some have given this as anti-libertarian commentary. But first of all, in analyzing this statement by Jesus it needs to be pointed out that it is easier for a camel to go through the eye of a needle than for any person whosoever to enter the Kingdom of God. But Jesus also said that ``The things which are impossible with men are possible with God.''\footnote{Luke 18:27.} It is standard Christian doctrine that it is impossible for anyone to enter the Kingdom of God on their own---that the only way in which anyone enters the Kingdom of God is through the saving grace of Jesus Christ alone.\footnote{See John 14:6.} Thus, the rich are by no means unique in this particular aspect. And so also, from this alone it cannot be claimed that Jesus had it in for rich people \emph{per se} more than any other group.

Second, when Jesus counseled this particular rich person to sell all that he had and distribute the proceeds to the poor, this was in fact an exceedingly libertarian thing for Jesus to advise this person. For this was not just any kind of rich person---this was a rich person of a particular type: a ``ruler,'' i.e., one who has some variety of command over an Earthly, mortal government. And thus, the riches that this particular rich person was in possession of had been obtained through extortion and theft, i.e., by the threat and force of arms and might---this particular ruler's opinion to the contrary\footnote{Luke 18:21.} not withstanding scrutiny: almost no rulers throughout history have ever regarded their wealth as having been obtained through stealing. As Augustine of Hippo wrote:

\begin{squote}
Justice being taken away, then, what are kingdoms but great robberies? For what are robberies themselves, but little kingdoms? The band itself is made up of men; it is ruled by the authority of a prince, it is knit together by the pact of the confederacy; the booty is divided by the law agreed on. If, by the admittance of abandoned men, this evil increases to such a degree that it holds places, fixes abodes, takes possession of cities, and subdues peoples, it assumes the more plainly the name of a kingdom, because the reality is now manifestly conferred on it, not by the removal of covetousness, but by the addition of impunity. Indeed, that was an apt and true reply which was given to Alexander the Great by a pirate who had been seized. For when that king had asked the man what he meant by keeping hostile possession of the sea, he answered with bold pride, ``What thou meanest by seizing the whole earth; but because I do it with a petty ship, I am called a robber, whilst thou who dost it with a great fleet art styled emperor.''\footnote{Augustine, \emph{De Civitate Dei}, ca. 413--426, English translation: \emph{The City of God}, Vols. 1--2 in Marcus Dods (Ed.), \emph{The Works of Aurelius Augustine, Bishop of Hippo: A New Translation} (Edinburgh, Scotland: T.~\&~T. Clark, 1871--1976), 15 vols. Tradition gives this pirate's name as Diomedes: see, e.g., the Huguenot tract by Stephanus Junius Brutus (a nom de plume, with likely candidates for authorship being either of the Monarchomachists Philippe de Mornay or Hubert Languet), \emph{Vindiciae contra tyrannos} (Basel: 1579). For a prior telling of this anecdote, see Marcus Tullius Cicero, \emph{De re publica} (54--51 B.C.), Book III.}
\end{squote}

Thus, when Jesus offered this counsel to this particular rich person, He was merely telling this person what any good libertarian would have said in the same situation---particularly a natural-rights libertarian such as a Rothbardian.

\section{Jesus Engaged in Conspicuous Consumption when He Could have Provided for the Poor Instead}
\label{sec:JesusEngagedInConspicuousConsumptionWhenHeCouldHaveProvidedForThePoorInstead}

Some have maintained---usually in an effort to make some larger political point---that Jesus was some sort of ascetic who was against individuals having material riches, especially when those material goods could be used to provide for the poor instead. Yet Jesus Himself engaged in conspicuous consumption when He could have provided for the poor instead:

\begin{squotation}
And when Jesus was in Bethany at the house of Simon the leper, a woman came to Him having an alabaster flask of very costly fragrant oil, and she poured it on His head as He sat at the table. But when His disciples saw it, they were indignant, saying, ``Why this waste? For this fragrant oil might have been sold for much and given to the poor.''

But when Jesus was aware of it, He said to them, ``Why do you trouble the woman? For she has done a good work for Me. For you have the poor with you always, but Me you do not have always. For in pouring this fragrant oil on My body, she did it for My burial. Assuredly, I say to you, wherever this gospel is preached in the whole world, what this woman has done will also be told as a memorial to her.''\footnote{Matthew 26:6--13. See also Mark 14:3--9; Luke 7:37,38; John 12:1--8.}
\end{squotation}

Yet here in this case of luxurious consumption on the part of Jesus is purely of ornamental value, i.e., of a purely aesthetic value---and a fleeting one at that! When Jesus's disciples complained about this ``waste'' Jesus told His disciples to stop bothering the woman about it! At the very least, this demonstrates the notion that Jesus was some sort of austere, principled ascetic to be an untenable one---and thus also, any attempt to make some larger political point out of such a notion is automatically moot.

As well, Paul had this to say as to one's ultimate responsibility in providing for others: ``For even when we were with you, we commanded you this: If anyone will not work, neither shall he eat.''\footnote{2~Thessalonians 3:10.}

\section{Jesus Has Called Us to Liberty---Yet Those Who Pay Taxes Are Not Free!}
\label{sec:JesusHasCalledUsToLibertyYetThoseWhoPayTaxesAreNotFree}

A Bible passage that is sometimes referenced by etatists to supposedly demonstrate that Jesus supported the paying of taxes---but which in actuality demonstrates the exact opposite---is in Matthew 17:24--27:

\begin{squotation}
When they had come to Capernaum, those who received the temple tax came to Peter and said, ``Does your Teacher not pay the temple tax?''

He said, ``Yes.''

And when he had come into the house, Jesus anticipated him, saying, ``What do you think, Simon? From whom do the kings of the earth take customs or taxes, from their sons or from strangers?''

Peter said to Him, ``From strangers.''

Jesus said to him, ``Then the sons are free. Nevertheless, lest we offend them, go to the sea, cast in a hook, and take the fish that comes up first. And when you have opened its mouth, you will find a piece of money; take that and give it to them for Me and you.''
\end{squotation}

As previously stated,\footnote{See Section \ref{sec:JesusOnTaxesNothingIsRightlyCaesarS} on page \pageref{TempleTax} of this article.} it appears that the only reason Jesus paid the temple tax (and by supernatural means at that) as told above in Matthew 17:24--27 was so as not to stir up trouble which would have interfered with the necessary fulfillment of Old Testament Scripture\footnote{See Psalms 41:9; 69:25; 109:8; Zechariah 11:12,13. See also Matthew 26:54,56; Mark 14:49; John 13:18--30; Acts 1:15--26.} and Jesus's previous prediction of His betrayal as told in Matthew 17:22---neither of which would have been fulfilled had Jesus not paid the tax and been arrested because of it. Jesus Himself supports this view when He said of it ``Nevertheless, lest we offend \mbox{them \ldots,''} which can also be translated ``But we don't want to cause trouble''\footnote{Contemporary English Version.}---at any rate, this comment by itself clearly demonstrates that Jesus was hardly enthusiastic about the prospect of paying taxes.

But moreover, Jesus said this after in effect saying that those who pay customs and taxes are not free.\footnote{See Matthew 17:25,26.} This is the necessary implication of this passage, for if the sons of the kings on Earth are free because they are exempt from paying taxes then this certainly implies that those who are required to pay taxes are therefore \emph{not} free on that account---either that or Jesus was merely being banal when He said this (which at least from the Christian's viewpoint is certainly not something Jesus was ever known for). Yet the fact that Jesus considers those who are required to pay taxes as being unfree is enough to conclusively demonstrate that Jesus is necessarily against taxes, as one of the main reasons Jesus came was to call us to liberty! Jesus said this Himself as recorded in Luke 4:16--21:

\begin{squotation}
So He came to Nazareth, where He had been brought up. And as His custom was, He went into the synagogue on the Sabbath day, and stood up to read. And He was handed the book of the prophet Isaiah. And when He had opened the book, He found the place where it was written:

\begin{sverse}
``The Spirit of the \textsc{Lord} is upon Me,\\
Because He has anointed Me\\
To preach the gospel to the poor;\\
He has sent Me to heal the brokenhearted,\\
To proclaim liberty to the captives\\
And recovery of sight to the blind,\\
To set at liberty those who are oppressed;\\
To proclaim the acceptable year of the \textsc{Lord}.''
\end{sverse}

Then He closed the book, and gave it back to the attendant and sat down. And the eyes of all who were in the synagogue were fixed on Him. And He began to say to them, ``Today this Scripture is fulfilled in your hearing.''
\end{squotation}

So here we have it: Jesus Himself said that He came to proclaim liberty to the captives and to set at liberty the oppressed---and yet Jesus also said that those who are required to pay taxes are not free!

Some may attempt to get around this glaring fact by pointing out that the word ``free'' in Matthew 17:26 is a translation of the Greek word \emph{eleutheroi} (\ibygr{eleuqeroi}), whereas the word ``liberty'' in Luke 4:18 is a translation of the Greek word \emph{aphesei} (\ibygr{afesei}). But \emph{eleutheroi} is an adjective form of the noun \emph{eleutheria} (\ibygr{e)leuqeri'a}), and means: freeborn, i.e., in a civil sense, one who is not a slave, or of one who ceases to be a slave, freed, manumitted; or at liberty, free, exempt, unrestrained, not bound by an obligation---and \emph{aphesei} means: release from bondage or imprisonment; forgiveness or pardon, i.e., remission of the penalty; or liberty. Thus, when used in the context above these two words are completely congruent in meaning with each other. As well, if one desires to go back further to the original Hebrew of Isaiah 61:1 which Luke 4:18 is quoting from, the word \emph{aphesei} is a translation of the Hebrew word \textcjheb{rwrd} (which roughly transliterates as \emph{darowr}) which is a noun that means: a flowing (as of myrrh), free run, or liberty. And so this word, too, is completely congruent in meaning with \emph{eleutheroi} when used in the above context. Indeed, the Greek Septuagint translates this Hebrew word in the above passage as \emph{aphesin} (\ibygr{a)'fesin}). \emph{Aphesei} and \emph{aphesin} are just different inflections of the same root word. Thus it cannot be honestly maintained that Jesus had in mind two separate meanings when he spoke the above words, as the only sensible meaning of these separate words are completely congruent with one another when used in their above context.

It might be pointed out by some that the New International Version translates the Greek word \emph{eleutheroi} in Matthew 17:26 as ``exempt.'' But this is a damning example of how some modern Bible translations have been bowdlerized in order to avoid inconvenient facts---particularly political ones---that are often found in the Bible. As was mentioned before, if indeed this were assumed to be the correct translation of this word, then for Jesus to make such a pointless comment would have been rather arid on His part---again, not something Jesus was ever known for, at least from the true Christian's perspective. The only meaning in which this comment by Jesus can be taken which actually makes any point whatsoever and avoids imputing empty talk to Him is for the Greek word \emph{eleutheroi} in Matthew 17:26 to be translated as ``free'' (or otherwise ``at liberty,'' etc.)---which is precisely how the King James Version and most other English Bible translations have handled this passage. Again, trying to avoid this most obvious and direct translation renders Jesus's comment here otiose.

As well, Paul and the original apostles understood that one of the main reasons Jesus came was to call us to liberty. Thus, Paul writes:

\begin{squote}
You were bought at a price; do not become slaves of men.\footnote{1~Corinthians 7:23.}
\end{squote}

\begin{squote}
For though I am free [\emph{eleutheros}] from all men, I have made myself a servant to all, that I might win the more; and to the Jews I became as a Jew, that I might win Jews; to those who are under the law, as under the law, that I might win those who are under the law; to those who are without law, as without law (not being without law toward God, but under law toward Christ), that I might win those who are without law; to the weak I became as weak, that I might win the weak. I have become all things to all men, that I might by all means save some. Now this I do for the gospel's sake, that I may be partaker of it with you.\footnote{1~Corinthians 9:19--23.}
\end{squote}

\begin{squote}
Now the Lord is the Spirit; and where the Spirit of the Lord is, there is liberty [\emph{eleutheria}].\footnote{2~Corinthians 3:17.}
\end{squote}

\begin{squote}
And because you are sons, God has sent forth the Spirit of His Son into your hearts, crying out, ``Abba, Father!'' Therefore you are no longer a slave but a son, and if a son, then an heir of God through Christ.\footnote{Galatians 4:6,7.}
\end{squote}

\begin{squote}
Stand fast therefore in the liberty [\emph{eleutheria}] by which Christ has made us free [\emph{eleutherosen}], and do not be entangled again with a yoke of bondage.\footnote{Galatians 5:1.}
\end{squote}

\begin{squote}
For you, brethren, have been called to liberty [\emph{eleutheria}]; only do not use liberty [\emph{eleutheria}] as an opportunity for the flesh, but through love serve one another. For all the law is fulfilled in one word, even in this: ``You shall love your neighbor as yourself.''\footnote{Galatians 5:13,14.}
\end{squote}

James writes:

\begin{squote}
But he who looks into the perfect law of liberty [\emph{eleutherias}] and continues in it, and is not a forgetful hearer but a doer of the work, this one will be blessed in what he does.\footnote{James 1:25.}
\end{squote}

\begin{squote}
So speak and so do as those who will be judged by the law of liberty [\emph{eleutherias}].\footnote{James 2:12.}
\end{squote}

And Peter writes:

\begin{squote}
Live as free [\emph{eleutheroi}] men, yet without using your freedom [\emph{eleutherian}] as a pretext for evil; but live as servants of God.\footnote{1~Peter 2:16, Revised Standard Version.}
\end{squote}

As was already observed,\footnote{See Section \ref{sec:SlavesObeyYourMasters} on page \pageref{EleutheriaDefinition} of this article.} the Greek noun \emph{eleutheria} is completely congruent in meaning with the English word ``liberty,'' i.e., as in freedom from slavery, independence, absence of external restraint, a negation of control or domination, freedom of access, etc. Some have contended that any demarcation of property ``restricts liberty,'' i.e., the liberty of others to use these resources, and so have maintained that the very concept of ``total liberty'' for everyone is an untenable one. But as Prof. Murray N. Rothbard has pointed out:

\begin{squote}
This criticism misuses the term ``liberty.'' Obviously, any property right infringes on others' ``freedom to steal.'' But we do not even need property rights to establish this ``limitation''; the existence of another \emph{person}, under a regime of liberty, restricts the ``liberty'' of others to assault him. Yet, by definition, liberty \emph{cannot} be restricted thereby, because liberty is defined as freedom to \emph{control what one owns} without molestation by others. ``Freedom to steal or assault'' would permit someone---the victim of stealth or assault---to be forcibly or fraudulently deprived of his person or property and would therefore violate the clause of total liberty: that \emph{every} man be free to do what he wills with his own. Doing what one wills with \emph{someone else's} own impairs the other person's liberty.\footnote{Murray N. Rothbard, \emph{Power and Market: Government and the Economy} (Kansas City: Sheed Andrews and McMeel, Inc., 1970), p. 242 \textless\url{http://web.archive.org/web/20050923192825/mises.org/rothbard/power&market.pdf}\textgreater , \textless\url{http://webcitation.org/5ve3w5w9a}\textgreater .}
\end{squote}

\section{Jesus Will Overthrow All the Governments of the World and Punish All the Rulers in the Time of His Judgement (i.e., His Second Coming)}
\label{sec:JesusWillOverthrowAllTheGovernmentsOfTheWorld}

In the above it was clearly demonstrated that the Earthly, mortal governments are firmly under the control of Satan---that it is Satan who is the true god and ruler over this perverted governmental world system wherein power-mad psychopaths rule over our existence and exempt themselves from every standard of decency which people would otherwise expect from any common stranger. Yet this diabolical, demonically-controlled government system is not to last forever. The Bible is quite clear and explicit in many passages as to what God's Judgement---i.e., the Second Coming of Christ---is to be about.

The Devil's false Christ---i.e., the Antichrist---will come to strengthen and empower government during the last days: cementing together for the first time in human history a world government---of which God will allow to continue for a short time.\footnote{See Revelation 17:9--18.} This world government will be the ultimate culmination of the very essence of everything which government represents: in short, it will be the most diabolical government which has ever existed, with mass-murder of the righteous on a massive scale.\footnote{See Revelation 20:4.} All the rulers of the Earth will whore themselves with this world government and be aligned against Jesus Christ during the final battle of Armageddon.\footnote{See Revelation 16:14; 17:2; 18:3,9; 19:19.}

Yet the coming of God's true Christ---Jesus Christ---is to be the exact opposite of Satan's Christ! Instead of strengthening government, Jesus Christ will come to abolish and utterly annihilate all the governments of the world: including all the rulers of those governments along with them!

As it is written in the Old Testament concerning the End-Times Judgement of God, i.e., Jesus's Second Coming:

\begin{sverse}
The Lord is at Your right hand;\\
He shall execute kings in the day of His wrath.\\
He shall judge among the nations,\\
He shall fill the places with dead bodies,\\
He shall execute the heads of many countries.\footnote{Psalms 110:5,6.}
\end{sverse}

And the above prophecy is mirrored by the prophet Isaiah:

\begin{sverse}
It shall come to pass in that day\\
That the Lord will punish on high the host of exalted ones,\\
And on the earth the kings of the earth.\\
They will be gathered together,\\
As prisoners are gathered in the pit,\\
And will be shut up in the prison;\\
After many days they will be punished.\footnote{Isaiah 24:21,22.}
\end{sverse}

This is quite amazing indeed when one realizes that the prophet Ezekiel foresaw this exact thing concerning God's End-Times Judgement---this time as it specifically concerned the rulers over Israel:

\begin{squotation}
And the word of the \textsc{Lord} came to me, saying, ``Son of man, prophesy against the shepherds of Israel, prophesy and say to them, `Thus says the Lord \textsc{God} to the shepherds: ``Woe to the shepherds of Israel who feed themselves! Should not the shepherds feed the flocks? You eat the fat and clothe yourselves with the wool; you slaughter the fatlings, but you do not feed the flock. The weak you have not strengthened, nor have you healed those who were sick, nor bound up the broken, nor brought back what was driven away, nor sought what was lost; but with force and cruelty you have ruled them. So they were scattered because there was no shepherd; and they became food for all the beasts of the field when they were scattered. My sheep wandered through all the mountains, and on every high hill; yes, My flock was scattered over the whole face of the earth, and no one was seeking or searching for them.''

``\thinspace`Therefore, you shepherds, hear the word of the \textsc{Lord}: ``As I live,'' says the Lord \textsc{God}, ``surely because My flock became a prey, and My flock became food for every beast of the field, because there was no shepherd, nor did My shepherds search for My flock, but the shepherds fed themselves and did not feed My flock''---therefore, O shepherds, hear the word of the \textsc{Lord}! Thus says the Lord \textsc{God}: ``Behold, I am against the shepherds, and I will require My flock at their hand; I will cause them to cease feeding the sheep, and the shepherds shall feed themselves no more; for I will deliver My flock from their mouths, that they may no longer be food for them.''\thinspace'\thinspace''\footnote{Ezekiel 34:1--10.}
\end{squotation}

Now obviously when God, speaking here to Ezekiel, refers to ``shepherds,'' He is using this as a metaphor for rulers, just as ``flock'' is a metaphor for the masses of people. Consider also the following passage spoken to the prophet Zechariah concerning God's End-Times Judgement: ``My anger is kindled against the shepherds,~/ And I will punish the goatherds.''\footnote{Zechariah 10:3.}

Now obviously again, God, speaking here to Zechariah---just as Ezekiel before him---is not talking about literal shepherds and goatherds, but is using these expressions as metaphors for rulers. Indeed, this is how the New Revised Standard Version translates it: ``My anger is hot against the shepherds, and I will punish the leaders\mbox{.\hspace{0.167em}\ldots''}

Thus, there is an amazing continuity within the Old Testament prophecies as to what God's End-Times Judgement is, at least in part, to consist of: the punishment of all the Earthly rulers and the abolition of all mortal rulerships! Can there be any doubt left in an honest, true Christian's mind as to just how much Jesus absolutely abhors and detests government? If there should be the slightest shred of doubt left in one's mind, then please, choose to walk in the clear light of liberty and let Paul slay---once and for all---that last misplaced sense of doubt!:

\begin{squote}
However, we speak wisdom among those who are mature, yet not the wisdom of this age, nor of the rulers of this age, who are coming to nothing. But we speak the wisdom of God in a mystery, the hidden wisdom which God ordained before the ages for our glory, which none of the rulers of this age knew; for had they known, they would not have crucified the Lord of glory.\footnote{1~Corinthians 2:6--8.}
\end{squote}

\begin{squote}
But each one in his own order: Christ the firstfruits, afterward those who are Christ's at His coming. Then comes the end, when He delivers the kingdom to God the Father, when He puts an end to all rule and all authority and power.\footnote{1~Corinthians 15:23,24.}
\end{squote}

How could it possibly be stated any clearer?! The governments of the Earth are not of God, they are of Satan, and Jesus will come to utterly destroy them \emph{all} during His Judgement!

As it is written:

\begin{squote}
And I saw the beast, the kings of the earth, and their armies, gathered together to make war against Him who sat on the horse and against His army. Then the beast was captured, and with him the false prophet who worked signs in his presence, by which he deceived those who received the mark of the beast and those who worshiped his image. These two were cast alive into the lake of fire burning with brimstone. And the rest were killed with the sword which proceeded from the mouth of Him who sat on the horse. And all the birds were filled with their flesh.\footnote{Revelation 19:19--21.}
\end{squote}

In the above passage from Revelation 19, the ``rest'' referred to being ``killed with the sword which proceeded from the mouth'' of Jesus in verse 21 are ``the kings of the earth, and their armies, gathered together to make war against Him who sat on the horse and against His army,'' which was previously referred to in verse 19.

And so it is found that from the Old Testament through the New Testament that there is a remarkable continuity and agreement as to what the fate of all the Earthly governments and all their rulers shall be during God's Judgement. And so also, this all demonstrates unmistakably just how much God is opposed to the ghastly, Satanical machination called government!

There can be no honest doubt: Jesus is an anarchist!

\section{God's People Are to Be Volunteers and Self-Rulers in the Kingdom of Christ}
\label{sec:GodSPeopleAreToBeVolunteersAndSelfRulersInTheKingdomOfChrist}

Some may object to the designation of Jesus as an anarchist---as some may counter, What about the Kingdom of Christ that is to be established after the Judgement? But as was pointed out in several places above, the ``Kingdom of Christ'' will in no sense be an actual government as they have existed on Earth and operated by mortals. For the Kingdom of Christ is to be the diametrically functional opposite of any government which has ever existed on Earth before. Thus, it is perfectly fine to refer to the ``Kingdom of Christ'' so long as one bears in mind that it has nothing whatsoever to do with any historical government that has ever existed. And so when it is said herein that ``Jesus is an anarchist,'' this is merely an objective designation as it refers to all Earthly, mortal governments, and all governments of their kind. People have been trained from birth by the Satanic, mortal governments to fear this word and to recoil from it, but it is used here only in its most objective sense.

It has been said above that the Kingdom of Christ is to be the functional opposite of any government which has ever existed before. What exactly is meant by this?

Well, to begin with, unlike all mortal governments, which compel people to support them whether they want to or not---in the form of taxes, etc.---the only thing which anyone can give to God which He does not already have is their voluntary love. God gives to all their very life, and God sustains all.\footnote{See Job 34:14,15; Acts 17:25.} The seeking of material possessions means nothing to God as He is what makes their very existence possible. Therefore taxes and their like will have no place in God's Kingdom, as God has no need for such material support, as do the mortal governments.

But God is always seeking our love: but true love cannot be forced from someone, real love can only be a voluntary process. Therefore there will be no compulsion on the part of God. As it is written in Psalm 110:3 concerning the establishment of Jesus's Kingdom:

\begin{sverse}
Your people shall be volunteers\\
In the day of Your power;\\
In the beauties of holiness, from the womb of the morning,\\
You have the dew of Your youth.
\end{sverse}

Thus the people of God's Kingdom shall be volunteers! How different indeed from all the mortal governments which compel people to support them through theft and extortion!

And in further elaboration of this, let us consider the following passage from Revelation 5:8--10:

\begin{squotation}
Now when He had taken the scroll, the four living creatures and the twenty-four elders fell down before the Lamb, each having a harp, and golden bowls full of incense, which are the prayers of the saints. And they sang a new song, saying:

\begin{sverse}
``You are worthy to take the scroll,\\
And to open its seals;\\
For You were slain,\\
And have redeemed us to God by Your blood\\
Out of every tribe and tongue and people and nation,\\
\textsuperscript{verse~10}And have made us kings and priests to our God;\\
And we shall reign on the earth.''\footnote{See also Revelation 1:6.}
\end{sverse}
\end{squotation}

Yet what exactly is verse 10 in the above passage talking about? If we righteous shall all be volunteers and all the impenitent workers of iniquity have been cast into the lake of fire, then who exactly is left for us to be king over and what exactly shall we be reigning over? \emph{Each other?} Does that make any sense?

Obviously the only \emph{who} for us to be kings over is our own persons and the only \emph{what} for us to reign over shall be our own domain. For the first time in history mankind will truly be free from the yoke of bondage---that Satanic world system of servitude in all of its many guises. For the first time ever we will be self-rulers and our homes truly will be our castles! We shall be complete and absolute sovereigns over our own lives!

Because it very much bears repeating, I will leave this section by citing what Paul had to say on this matter one more time, for he said it as well and as plainly as it could possibly be stated:

\begin{squote}
But each one in his own order: Christ the firstfruits, afterward those who are Christ's at His coming. Then comes the end, when He delivers the kingdom to God the Father, when He puts an end to all rule and all authority and power.\footnote{1~Corinthians 15:23,24.}
\end{squote}

Amen.

\section{Closing Remarks}
\label{sec:ClosingRemarks}

\epigraph{The Christian ideal has not been tried and found wanting. It has been found difficult; and left untried.}{Gilbert Keith Chesterton\\
\footnotesize Part 1, Chapter 5: ``The Unfinished Temple,'' in \emph{What's Wrong With the World} (New York: Dodd, Mead and Company, 1910)}

\noindent
In all of my research into Jesus Christ I have discovered that He is nothing if not a perfectly consistent libertarian, at least as it concerns His political ethic. I could come across not one instance of Him contradicting this position, either in word or in action. I can't say that I was really surprised by this, although I suppose to many it may be surprising to learn this. For one thing, when Jesus gave the Golden Rule as the ultimate social ethic,\footnote{See Matthew 7:12; Luke 6:31.} it's clear that He actually meant it. Yet, as was demonstrated above,\footnote{See Section \ref{sec:TheGoldenRuleUnavoidablyResultsInAnarchism} on page \pageref{sec:TheGoldenRuleUnavoidablyResultsInAnarchism} of this article.} this ethic is just a different formulation of the libertarian Nonaggression Principle, at least as a political ethic. As a strictly political ethic it is actually completely congruent with the libertarian Nonaggression Principle, in that as political ethics they actually prohibit the same activity: i.e., aggression against people's just property---and ultimately all just property titles can (1) be traced back by way of voluntary transactions (which would thus be consistent with the Golden Rule) to the homesteading of unused resources; or (2) in the case in which such resources were expropriated from (or abandoned by) a just owner and the just owner or his heir(s) can no longer be identified or are deceased, where the first nonaggressor possesses the resource (which can then be considered another form of homesteading).

What I have shown above is that Jesus has called us to liberty, and that liberty and Christ's message are incompatible with government. Indeed, governments throughout history have been the most demonic force to ever exist on Earth. We need not lament their passing, but instead look forward to it.

Before I leave you, there exists a couple of other points that need to be mentioned as to what the importance of this message is:

To start with, as Christians how can we be attentive to the cries of the oppressed if we don't even recognize the oppressor? How can we comfort and give aid to someone if we don't even recognize them as a victim? We are liable to be obtuse and uncaring to those who have been unjustly wronged by this Satanic world system if we don't even recognize the main instrument of Satan's power on this Earth. So that is first and foremost: by realizing and understanding the truth as to the diabolical nature of government one will thereby have gained back part of one's humanity which this Satanic world system has worked so hard in making people oblivious to. One need only watch some of the old Nazi propaganda films of thousands of German youths goose-stepping in unison to realize just how effective this demonic world system can sometimes be in stripping people of their humanity.

Second, according to the Bible, it makes a difference as to when Jesus's Second Coming will occur depending on our actions in being able to raise the awareness of the world's population. While although I mentioned \'{E}tienne de La Bo\'{e}tie in the introduction and pointed out that if a critical mass of the population could come to understand and accept the truth as to the true nature of governments that it would be enough to topple them, this is ultimately true because it would in this case hasten the coming of Jesus Christ! Thus, Peter wrote about Christians being able to hasten the coming of Christ:

\begin{squote}
Therefore, since all these things will be dissolved, what manner of persons ought you to be in holy conduct and godliness, looking for and hastening the coming of the day of God, because of which the heavens will be dissolved, being on fire, and the elements will melt with fervent heat?\footnote{2~Peter 3:11,12.}
\end{squote}

And another extremely important reason for this message presented herein has already been touched on in Section \ref{sec:JesusWillOverthrowAllTheGovernmentsOfTheWorld} above. The Bible tells of a massive End-Times deception perpetrated by the Devil upon the masses in the form of the Antichrist. Although if one understands what the coming of God's real Christ is to be about---as Paul puts it ``Then comes the end, when He delivers the kingdom to God the Father, when He puts an end to all rule and all authority and power''\footnote{1~Corinthians 15:24.}---then it will be impossible for one to be deceived by the Antichrist, as the Antichrist will come to strengthen government, not to abolish it. Some Christians mistakenly believe that so long as one accepts a person called ``Jesus'' as their Lord and Savior then they will have eternal salvation. Yet there will be many people in the End-Times Judgement who will consider themselves to be good Christians worshiping the true Second Coming of Jesus Christ, and yet in doing so they will have condemned themselves! The Antichrist will present himself as being the Second Coming of Jesus! But Jesus said, ``I am the way, the truth, and the life. No one comes to the Father except through Me''!\footnote{John 14:6.} Thus if one worships a lie in place of the truth then the fact that one will have called this lie by the name of ``Jesus'' will be of no help! In fact, to do so is blasphemy! In order for one to really worship Jesus one first has to know what the truth of Jesus is about. And that, my friends, is the ultimate purpose of this document: that people may come to know the real Jesus. And what Jesus Christ is about is \emph{liberty}---at least as politics is concerned.

But lastly, many unjust government actions have been supported by self-professed Christians, such as with Prohibition and the War on Drugs, even though such nefarious laws are completely unjustifiable from a Biblical perspective and indeed very anti-Christian in the most literal sense of the word. As well, such government actions as taxes are also completely anti-Christian. Thus, in clearly demonstrating how Jesus was nothing if not a perfectly consistent libertarian---at least as it concerned His political ethic---from this Christians can get a clear picture as to what their objectives should be as it concerns such matters, instead of ``giving heed to deceiving spirits and doctrines of demons'' as Paul put it.\footnote{1~Timothy 4:1.} I dread to think how many young men have been raped in the U.S. prison system because they had violated some make-believe ``crime'' against using or selling certain pharmaceuticals---that aggress against no one---which people calling themselves Christians had supported. As Christians, we need to be aware of the tricks Satan has used throughout history to get people to support his empowerment. We need to be above all the pettiness and walk in the clear light of liberty which Jesus commanded us and declare everyone to be a sovereign over their own domain, unless they should violate another's right of the same.

\appendix
\appendixpage

\section{Articles Everyone Should Be Familiar With}
\label{sec:AppendixArticlesEveryoneShouldBeFamiliarWith}

\begin{enumerate}
\small
\item Prof. Murray N. Rothbard, ``The Anatomy of the State,'' \emph{Rampart Journal of Individualist Thought}, Vol. 1, No. 2 (Summer 1965), pp. 1--24. Reprinted in a collection of some of Rothbard's articles, \emph{Egalitarianism As a Revolt Against Nature and Other Essays} (Washington, D.C.: Libertarian Review Press, 1974) \textless\url{http://mises.org/books/egalitarianism.pdf}\textgreater , \textless\url{http://webcitation.org/5ve3r05ti}\textgreater .

\item Prof. Murray N. Rothbard, ``Defense Services on the Free Market,'' Chapter~1 from \emph{Power and Market: Government and the Economy} (Kansas City: Sheed Andrews and McMeel, Inc., 1970), pp. 1--9 \textless\url{http://web.archive.org/web/20050923192825/mises.org/rothbard/power&market.pdf}\textgreater , \textless\url{http://webcitation.org/5ve3w5w9a}\textgreater .

\item Prof. Hans-Hermann Hoppe, ``The Private Production of Defense,'' \emph{Journal of Libertarian Studies}, Vol. 14, No. 1 (Winter 1998--1999), pp. 27--52 \textless\url{http://mises.org/journals/jls/14_1/14_1_2.pdf}\textgreater , \textless\url{http://webcitation.org/5ve41VasQ}\textgreater .

\item Prof. Hans-Hermann Hoppe, ``Fallacies of the Public Goods Theory and the Production of Security,'' \emph{Journal of Libertarian Studies}, Vol. 9, No. 1 (Winter 1989), pp. 27--46 \textless\url{http://mises.org/journals/jls/9_1/9_1_2.pdf}\textgreater , \textless\url{http://webcitation.org/5ve485kNf}\textgreater .

\item Prof. Frank J. Tipler, ``The structure of the world from pure numbers,'' \emph{Reports on Progress in Physics}, Vol. 68, No. 4 (April 2005), pp. 897--964, \href{http://dx.doi.org/10.1088/0034-4885/68/4/R04}{\nolinkurl{doi:10.1088/0034-4885/68/4/R04}} \textless\url{http://math.tulane.edu/~tipler/theoryofeverything.pdf}\textgreater , \textless\url{http://webcitation.org/5nx3CxKm0}\textgreater . Also released as ``Feynman--Weinberg Quantum Gravity and the Extended Standard Model as a Theory of Everything,'' arXiv:0704.3276, April 24, 2007 \textless\url{http://arxiv.org/abs/0704.3276}\textgreater .
\end{enumerate}

\section{Biography of the Author}
\label{sec:BiographyOfTheAuthor}

Born in Austin, Texas and raised in the Leander, Texas hill country, James Redford is a born-again Christian who was converted from atheism by a direct revelation from Jesus Christ. He is a scientific rationalist who concludes that the Omega Point (i.e., the physicists' technical term for God) and the Feynman--DeWitt--Weinberg quantum gravity\slash Standard Model Theory of Everything (TOE) is an unavoidable result of the known laws of physics. His website is the following:

\begin{itemize}
\small
\item \emph{Theophysics: God Is the Ultimate Physicist} \textless\url{http://theophysics.chimehost.net}\textgreater  , \textless\url{http://theophysics.host56.com}\textgreater , \textless\url{http://theophysics.ifastnet.com}\textgreater .
\end{itemize}
\end{document}